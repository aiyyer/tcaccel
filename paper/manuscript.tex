%% Copernicus Publications Manuscript Preparation Template for LaTeX Submissions
%% ---------------------------------
%% This template should be used for copernicus.cls
%% The class file and some style files are bundled in the Copernicus Latex Package, which can be downloaded from the different journal webpages.
%% For further assistance please contact Copernicus Publications at: production@copernicus.org
%% https://publications.copernicus.org/for_authors/manuscript_preparation.html


%% Please use the following documentclass and journal abbreviations for preprints and final revised papers.

%% 2-column papers and preprints
\documentclass[wcd,manuscript]{copernicus}
% Weather and Climate Dynamics (wcd)


%% \usepackage commands included in the copernicus.cls:
%\usepackage[english]{babel}
%\usepackage{tabularx}
%\usepackage{cancel}
%\usepackage{multirow}
%\usepackage{supertabular}
%\usepackage{algorithmic}
%\usepackage{algorithm}
%\usepackage{amsthm}
%\usepackage{float}
%\usepackage{subfig}
%\usepackage{rotating}

\begin{document}

\title{Acceleration of Tropical Cyclones As a Proxy For Extratropical Interactions: Synoptic Scale Patterns and Long-Term Trends}


% \Author[1]{given_name}{surname}

\Author[1,*]{Anantha}{Aiyyer}
\Author[1,*]{Terrell}{Wade}

\affil[1]{Department of Marine, Earth and Atmospheric Sciences, North Carolina State University}


%% The [] brackets identify the author with the corresponding affiliation. 1, 2, 3, etc. should be inserted.

%% If an author is deceased, please mark the respective author name(s) with a dagger, e.g. "\Author[2,$\dag$]{Anton}{Aman}", and add a further "\affil[$\dag$]{deceased, 1 July 2019}".

%% If authors contributed equally, please mark the respective author names with an asterisk, e.g. "\Author[2,*]{Anton}{Aman}" and "\Author[3,*]{Bradley}{Bman}" and add a further affiliation: "\affil[*]{These authors contributed equally to this work.}".


\correspondence{A. Aiyyer (aaiyyer@ncsu.edu)}

\runningtitle{Tropical Cyclone Acceleration}

\runningauthor{Aiyyer and Wade}



\received{}
\pubdiscuss{} %% only important for two-stage journals
\revised{}
\accepted{}
\published{}

%% These dates will be inserted by Copernicus Publications during the typesetting process.


\firstpage{1}

\maketitle


\begin{abstract}

  change tense to present...

  
  It is well known that rapid changes in tropical cyclone (TC) motion occur during interaction with extratropical wavetrains. While the forward speed of TCs has received much attention in the past, acceleration has not. Using a large data sample, we formally examined the synoptic-scale patterns associated with  tangential and normal components of TC acceleration. During rapid acceleration, a developing wavepacket in the extratropics is present and the TC ?????? with the TC straddled by an upstream trough and downstream ridge. This pattern is consistent with the process of extratropical transition (ET) of TCs. In contrast, during near-zero or rapid deceleration, the extratropical wavepacket remains poleward, with a ridge  located directly north of the storm center.  On average, the tangential acceleration peaks about 18 hours prior to the ET designation. The curvature acceleration increases before ET and stabilizes thereafter.
%
 We also find a robust decrease in the annually averaged values of rapid acceleration over the past 38 years within the latitude band most commonly associated with ET over the Atlantic. Consequently, we hypothesize that trends in rapid acceleration can be used as an independent metric to assess changes in the extratropical stormtrack activity that mediate ET.
\end{abstract}


%\copyrightstatement{TEXT}


\introduction

The motion of a tropical cyclone (TC) is known to be  governed by the large-scale environment surrounding it \citep[e.g.][]{Hodanish1993}. It was recognized early on that the forward speed of a TC can be closely approximated by the surrounding wind field \citep{E2018}.  It is also understood that a TC is not an isolated vortex that is passively carried by a current. The background environment also has attendant gradients of potential vorticity, moisture, and wind shear. A TC actively responds to these external stimuli. Consequently, this impacts not only the motion and intensity of the TC, but also the the large scale flow. For instance, the generation of the $\beta$-\emph{gyres} -- that influence TC motion -- is a response to the background potential vorticity gradient (citation). Similarly, vertical wind shear can modulate the intensity of the TC via ventilation of low entropy air into the storm core (citation) . This, in turn,  can also impact subsequent TC motion. On the other hand, a TC can influence the large-scale flow by exciting  waves in the extratropical stormtrack, leading to rapid downstream development and rearrangement of the flow. Thus, it can be contended that a TC is always interacting with its environment and the interaction is, in part, manifested in its motion.

The aspects of TC motion that are particularly important for operational forecasts are track and forward speed. The track garners much attention for obvious reasons -- it  informs  potential locations that will be affected by a storm. TC forward speed is an important parameter that impacts storm surge, intensity changes and precipitation. There is a large body of published literature dealing with various research and operational problems related to TC tracks and speed. They are extensively documented in several recent reviews \citep[e.g.,][]{E2018}. The focus of this paper is the acceleration and deceleration of TCs. While Forecasters have long known that the TCs often speed up when approaching an extratropical wavetrain, relatively less attention appears to have been devoted in the research literature to details of TC acceleration. 

When TCs move poleward, they encroach upon the extartropical stormtrack. This leads to an interaction between the TCs and the baroclinic eddies of the stormtrack. The outcome of this interaction is varied. Some TCs weaken and dissipate while others may strengthen and retain their tropical nature for some more time. In the Atlantic, approximately ?\% of storms undergo  extratropical transitoin (ET) - a complex process that  


%As noted by \cite{E2018}, steady progress has been made in our understanding of TC motion and this is reflected in our ability to forecast their tracks.




%To our knowledge, relatively few studies have examined the synoptic patterns associated with various thresholds of TC acceleration. 


\section{Data}
We used the IBTrACS v4 \citep{KnappIBTRaCS} data for track information. We did not apply a threshold for maximum sustained winds, but discarded all storms that lasted less than three days. Thus our definition of a TC in this paper includes a few tropical depressions that lasted greater than 3 days. 

{\bf Cite Kossin and the other paper here.}

We focused on the years 1966--2019, spanning the \emph{satellite era} for the North Atlantic and Pacific basins (e.g.. Beven et al. 2003; Landea 2007; Vecchi and Knutson), for which the TC data is deemed relatively more reliable as compared to previous years. Selected atmospheric fields were taken from the the European Center for Medium Range Weather Forecasting's (ECMWF) ERA-Interim (ERAi) reanalysis \citep{DUS11} for the period 1981--2016.


 As our intent here was to examine the motion of tropical storms, in many of our analysis, we excluded instances when a given storm became extratropical. To ascertain whether a tropical storm underwent ET, we used the \emph{nature} designation in the IBTraCS database that relies on the judgement of the forecasters from one or more agencies responsible for the ocean basin.  Admittedly, there will be some subjectivity in this designation. An alternative would be to employ a metric such as the cyclone phase space \citep{Hart2003}, usually calculated using meteorological fields from global numerical weather prediction models (e.g., reanalysis). As we wish to be independent of modeled products to characterize the storm, we choose to use the forecaster designated storm nature. Furthermore, \cite{BCSEH2019a} found that the phase space calculation was sensitive to the choice of the reanalysis model. Occasionally, the nature of the storm is not recorded (NR) or, if there is an inconsistency among the agencies, it is designated as mixed (MX). In the north Atlantic, only a small fraction of track data ($\approx 0.5\%$) in the IBTrACs is designated as NR or MX. However, the fraction is higher in the North Pacific ($\approx 14\%$). As such, for this basin, we include all storm positions, irrespective of nature designation. We return to this point in a subsequent sections where it could introduce some uncertainty in our conclusion for this basin. For this reason, we focus on Atlantic TCs in this paper, and provide selected results for western North Pacific storms in Appendix B.
 
 
\section{Tangential and curvature acceleration}
The acceleration of a hypothetical fluid element moving with the center of a TC can be written as:
%
\begin{equation}
    \mathbf{a} = \frac{dV}{dt} \mathbf{\hat{s}} + \frac{V^2}{R} \mathbf{\hat{n}} 
    \label{eq:acc}
\end{equation}
%
where $V$ is the forward speed and $R$ is the radius of curvature of the track at a given location. Here, $\mathbf{\hat{s}}$ and $\mathbf{\hat{n}}$ are orthogonal unit vectors in the so-called \emph{natural coordinate} system. The former is directed along the TC motion. The latter is directed along the radius of curvature of the track.  The first term in eq. \ref{eq:acc} is the tangential acceleration and the second is the curvature or normal acceleration. The speed $V_j$ at any track location (given by index j)  is calculated as:
%
\begin{equation}
V_j = \frac{(D_{j,j-1} +  D_{j,j+1})}{\delta t}
\end{equation}
%
where $D$ refers to the distance between the two consecutive points, indexed as shown above, along the track. Since we used 3-hourly reports from the IBtracs, $\delta t = 6$ h. The tangential acceleration was calculated from the speed using centered differences. 


To calculate the curvature acceleration, it is necessary to first determine the radius of curvature, R. A standard approach to calculating R, given a set of discrete points along a curve -- in our case, a TC track -- is to fit a circle through three consecutive points. For a curved line on a sphere, it can be shown that: 


\begin{subequations}
\renewcommand{\theequation}{\theparentequation.\arabic{equation}}
\begin{equation}
R  = R_e sin^{-1} {\left(\sqrt{\frac{2 d_{12} d_{13} d_{23}} {(d_{12} +d_{13} +d_{23})^2 -2(d_{12}^2 +d_{13}^2 +d_{23}^2)}}\right)}
\end{equation}
%
where $R$ is the radius of curvature, $R_e$ is the radius of the Earth, and the $d$ terms are expressed as follows:
%
\begin{gather}
 d_{12} = 1-(\cos{T_1}\cos{T_2}\cos{(N_2-N_1)} + \sin{T_1}\sin{T_2}) \\
 d_{13} = 1-(\cos{T_1}\cos{T_3}\cos{(N_3-N_1)} + \sin{T_1}\sin{T_3}) \\
 d_{23} = 1-(\cos{T_2}\cos{T_3}\cos{(N_3-N_2)} + \sin{T_2}\sin{T_3})
\end{gather}
\end{subequations}
%
where, $T_1$, $T_2$, and $T_3$ are the latitudes of the 3 points while $N_1$, $N_2$, and $N_3$ are the longitudes. The center of the circle is given by the coordinates:
%
\begin{subequations}
\renewcommand{\theequation}{\theparentequation.\arabic{equation}}
\begin{gather}
\tan{\mathbf{T}} = \pm {\frac{\cos{T_1}\cos{T_2}\sin{(N_2-N_1)} + \cos{T_1}\cos{T_3}\sin{(N_1-N_3)} + \cos{T_2}\cos{T_3}\sin{(N_3-N_2)}} {\sqrt{\alpha^2+\beta^2}}} \\ \tan{\mathbf{N}} = -{\frac{\alpha}{\beta}}
\end{gather}
%
Where $\mathbf{T}$ and $\mathbf{N}$ are the latitude and longitude, respectively of the circle's center, and $\alpha$ and $\beta$ are obtained using:

\begin{gather}
\alpha = \cos{T_1}(\sin{T_2}-\sin{T_3})\cos{N_1} + \cos{T_2}(\sin{T_3} -\sin{T_1})\cos{N_2} +\cos{T_3}(\sin{T_1}-\sin{T_2})\cos{N_3} \\
\beta = \cos{T_1}(\sin{T_2}-\sin{T_3})\sin{N_1} + \cos{T_2}(\sin{T_3}-\sin{T_1}) \sin{N_2} +\cos{T_3}(\sin{T_1}-\sin{T_2})\sin{N_3}
\end{gather}
\end{subequations}

While the expressions above were derived independently for this work, we make no claim of originality since they are based on elementary principles of geometry. Figure \ref{fig:track} shows some examples of the radius of curvature calculation for hurricanes Katrina (2005) and Irma (2017). 

\section{Basic Statistics}

During 1966--2019, we identified 689 storms in the North Atlantic that met the 2-day threshold. Figure \ref{fig:distNA} shows some basic statistics for speed and accelerations as a function of latitude for Atlantic storms. TC locations were sorted in $10^\circ$-wide latitude bins for this and all instances classified as NR or ET were excluded. Table \ref{tab:distNA} provides the same data. 


The average speed of all North Atlantic TCs (including over-land) is about 21 km/hr. As expected, TC speed clearly varies with latitude. It is lower in the subtropics as compared to other latitudes. It increases sharply in the vicinity of $40^{o}$ N. The tangential acceleration, as defined in eq. \ref{eq:acc}, can be positive or negative. The mean and median tangential acceleration remains near-zero equatorward of $30^\circ$N. This suggests that TCs in this region tend to translate steadily, or are equally likely to accelerate and decelerate. The tangential acceleration is substantially positive poleward of $30^\circ$N. TCs in these latitudes are subject to rapid acceleration. For example, the mean tangential acceleration in the $30^\circ-40^\circ$N latitude band is about 5.0  km/hr day$^{-1}$.  The curvature acceleration, by our definition, takes only positive values and, steadily increases with latitude. The distributions of TC speed and tangential acceleration are relatively symmetric about the median value as compared to the curvature acceleration.

Figure \ref{fig:distNAMonth} shows the distribution of mean tangential and curvature accelerations within the latitude band 20--40$^oN$ as a function of month. The numbers on top of the bars show the percentage of the total number of track points over all months. As expected, September shows the highest percentage of track points, followed by August and October. During the peak Atlantic hurricane season, the magnitude of the mean tangential and curvature accelerations increase steadily from July--November. 

%The corresponding information for western North Pacific for 1732 storms that met the 2-day threshold is presented in the appendix (Figure \ref{fig:distWP} and Table \ref{tab:distWP}). In this case, we consider all track points, including NR and ET for reasons discussed in section 2. The basic distribution of speed and accelerations are similar to the North Atlantic except that the storms, on average, were found to have negative tangential acceleration poleward of 40$^\circ$N. %In essence, we draw a distinction between the propagation of extratropical and tropical systems. We reason that an extratropical cyclone embedded in the stormtrack is subject to baroclinic development and dynamics of Rossby wavepackets \citep[e.g.][]{K2019}.  In contrast, a tropical system is primarily subject to steering by the background flow. \cite The importance of this distinction will be emphasized further in a subsequent section.




\section{Ensemble average flow for extremes in acceleration}

We now examine the flow pattern associated with extremes in tangential and curvature acceleration of tropical storms. For this, storm-relative ensemble average fields are constructed uisng ERA-interim gridded reanalysis over the period 1980--2015. We used the following method.

\begin{itemize}
    \item All TC track locations were binned into 10$^\circ$ wide latitude strips (e.g.,  10-20$^\circ$ N, 20-30$^\circ$ N, 30-40$^\circ$ N). Instances where storms were classified as ET or NR were excluded. For brevity, only results for the 30-40$^\circ$ N bin are discussed here. A total of 
    3480 track points were identified for this latitude bin. Note that a particular TC could appear more than once in a latitude bin at different times. 
    
    \item The tangential and curvature acceleration of all TCs within each latitude bin were sorted in an ascending order.

    \item The accelerations in each bin were separated into two categories based on whether they fell within: (a) The top 90th percentile of all sorted values; or (b) The bottom 10th percentile of all sorted values.
    
    \item For tangential acceleration, with both positive and negative values, we interpret these two categories as rapid acceleration and rapid deceleration respectively.
    
    \item For curvature acceleration, with only positive values, we interpret these two categories as rapid acceleration and near-zero acceleration respectively.

    
    \item For each category, we computed an ensemble average composite field of the geopotential field at selected levels. The composites were calculated after shifting the grids such that the center of all storms were coincident. The centroid position of storms was used for the composite storm center, and the corresponding time was denoted as Day-0. Lag-composites were created by shifting the dates backward and forward relative to Day-0.

    \item For anomaly fields, we subtracted long-term synoptic climatology form the total field. The climatology was calculated for each day of the year by averaging data for that day over the years 1980-2015. This was followed by a 7-day running mean smoother. To account for diurnal fluctuations, the daily climatology calculated for each available synoptic hour in the ERA-interim data (00, 06, 12 and 18 UTC).
    
    
    \item Statistical significance of anomaly fields shown in this section were evaluated by comparing them against 1000 composites created by randomly drawing dates from the period July--October, 1980--2015. The sample size for each composite was 350 for the 30-40$^\circ$ N bin. A two-tailed significance was evaluated at the 95\% confidence level with the null hypothesis being that the anomalies could have resulted from a random draw.
    
\end{itemize}

\subsection{Tangential Acceleration}


\subsubsection{Day 0 (reference day) composite}

Figure \ref{fig:tangComp} shows  storm-centered composite geopotential heights (thick orange lines) and their anomalies (color shaded) for all Atlantic TCs located within 30-40$^\circ$ N. Two categories of tangential acceleration are shown: rapid acceleration (left column) and rapid deceleration (right column).   The  fields are shown at three levels -- 300 hPa, 500 hPa and 850 hPa. In each panel, the anomalous 1000 hPa geopotential is shown using thinner black contours. It highlights the composite TC and the surface development within the extratropical stormtrack. 

The ensemble average for rapid tangential acceleration  (left column of Fig. \ref{fig:tangComp}) shows the composite TC interacting with a well defined extratropical wavepacket. The TC is straddled by an upstream trough and a downstream ridge. At 500 hPa, the geopotential anomalies of the TC and the upstream trough are close and, consequently, appear to be connected. This yields a negative tilt in the horizontal and indicates the onset of cyclonic wrap-up of the trough around the TC. The 1000-hPa geopotential anomaly field is dominated by the composite TC. It also shows the relatively weaker near-surface cyclones and anticyclones of the extratropical stormtrack. The entire wavepacket shows upshear tilt of geopotential anomalies with height, indicating baroclinic growth. This arrangement of the TC and the extratropical wavepacket is consistent with the synoptic-scale flow that is typically associated with ET \citep[e.g.,][]{KHE2000,MGY2003,RJD2008,RJ2014,K2019}. At this point, all storms in the ensemble were still classified as \emph{tropical}. Thus, we interpret this composite as pre-ET completion state. 

The ensemble average for rapid tangential deceleration cases (right column of Fig. \ref{fig:tangComp}) shows an entirely different synoptic-scale pattern. The extratropical wavepacket is substantially poleward, with a ridge immediately north of the composite TC. The geopotential anomalies of the extratropical wavepacket and the composite TC appear to be distinct at all three levels, with no evidence of merger. The prominent synoptic structure is the cyclone-anticyclone dipole formed by the TC and the extratropical ridge.

%------------------------------------------------------------------------------------------
% 



\subsubsection{Lag Composites}
To get a sense of the temporal evolution of the entire system, we show lag composites for Day-2 to Day+2 in Fig. \ref{fig:lagtang}. As in the previous figure, the two categories of acceleration are arranged in the respective columns. The rows now show  500-hPa geopotential height (thick contours) and anomalies (color shaded). In each panel, the corresponding 1000-hPa geopotential height anomalies are shown by thin black contours. 

The ensemble average for rapid tangential acceleration (left column of Fig. \ref{fig:lagtang}) shows a TC moving rapidly towards an extratropical wavepacket. At day-2, the TC circulation is relatively symmetric as depicted by the contours of 1000 hPa geopotential anomalies. The downstream extratropical ridge is prominent, but the upstream trough is much weaker at this time. On Day-1, the entire extratropical wavepacket has amplified and the 500 hPa geopotential anomalies of the TC and a developing upstream trough have merged. This process continues through Day 0. By Day+1, the composite storm has moved further poleward and eastward and is now located between the upper-level upstream trough and downstream ridge in a position that is optimal for further baroclinic development. The 1000 hPa geopotential field is now asymmetric with a characteristic signal of a cold front. 


The picture that is evident from these 500-hPa composite fields is that, over the course of the 4 days, downstream ridge-trough couplet amplifies while simultaneously propagating eastward. The upstream trough cyclonically wraps around the TC and the two have merged by Day 1. The geopotential gradient poleward of the storm is also enhanced, indicating a strengthening jet streak. These features are consistent the process of ET \citep[e.g.,][]{K2019}. The poleward moving TC may either interact with an existing wavepacket or perturb the extratropical flow and excite a Rossby wavepacket that disperses energy downstream \citep[e.g.,][]{RJ2014}. The outflow of the TC is a source of low potential vorticity (PV) air that further reinforces the downstream ridge \citep[e.g.,][]{RJD2008}. To further illustrate the interaction, the tracks of the TC and the 500-hPa ridge of the extratropical wavepacket  are presented in Fig. \ref{fig:tang_track}a. It can be clearly seen that the TC merges with the  extratropical stormtrack. Furthermore, the 500 hPa ridge has rapidly moved downstream during the 4 day period, indicating a very progressive pattern. The eastward phase speed of the extratropical wavepacket,  as inferred from the track of the ridge, is $\approx 7$ ms$^{-1}$. The TC speed, averaged over the same 4-day period,  is $\approx 6$ ms$^{-1}$. The close correspondence between the two and the  merger of the tracks further supports the notion that the synoptic-scale evolution during rapid acceleration cases is consistent with the canonical pattern associated with ET.

On the other hand, during rapid deceleration  (right column Fig. \ref{fig:lagtang}), the composite TC remains equatorward of the extratropical wavepacket and maintains a nearly symmetric structure throughout the period.  At all times, the arrangement of the TC and the extratropical ridge directly poleward is akin to a vortex dipole. The extratropical wavepacket is not as progressive as in the rapid acceleration case. This is seen clearly from the tracks of the TC and the ridge (Fig. \ref{fig:tang_track}b). The phase speed of the extratropical wavepacket is $\approx 3$ ms$^{-1}$ while the TC speed is $\approx 1.5$ ms$^{-1}$.  The phasing of the TC and the extratropical wavepacket has led to the formation of a cyclone-anticyclone vortex dipole. While this structure is not consistent with a typical ET process, the TC and the extratropical wavepacket are certainly interacting. If viewed from a potential vorticity (PV) perspective, the induced flow from the cyclonic vortex (TC) will be westward within the poleward anticyclonic vortex (ridge). Similarly, the induced flow from the ridge will also be westward at the location of the TC. The combined effect will be a mutual slowdown as reflected in the speeds estimated from Fig. \ref{fig:tang_track}b. 



%wavepacket remain nearly stationary, with little evidence of downstream development.  Thus, the near zero acceleration cases are not likely to be involved in ET. The rapid deceleration composite (right column Fig. \ref{fig:lagCompZ}) also shows that the extratropical wavepacket is poleward of the TC. However, by Day+1, there is evidence of amplification of the downstream ridge-trough couplet indicating development in the extratropical stormtrack related to the interaction with the TC.  This indicates a likelihood of ET at a later time.



\subsection{Curvature Acceleration}


\subsubsection{Day 0 (reference day) composite}

As in the  previous section, Figure \ref{fig:curvComp} shows storm-centered ensemble averages for two categories of curvature acceleration. The composite for rapid acceleration (left column) shows a TC that is primarily interacting with an extratropical ridge that is poleward and downstream of it. The upstream trough in the extratropics is weaker and farther westward as compared with the rapid tangential acceleration (Fig. \ref{fig:curvComp}a). Furthermore, instead of the upstream trough wrapping cyclonically, in this case,  we see the downstream ridge wrapping anticyclonically around the TC. The composite for near-zero curvature acceleration (right column) is quite similar to the composite for rapid tangential deceleration (Fig. \ref{fig:curvComp}d--f). The extratropical wavepacket is poleward and the TC-ridge system appears as a vortex dipole. 

\subsubsection{Lag Composites}
The temporal evolution of the entire system for the two categories of curvature acceleration is shown in Fig. \ref{fig:lagcurv}. For rapid curvature acceleration, we see a TC that is moving poleward towards an extratropical ridge. During the subsequent days, the ridge moves eastward initially and begins to wrap around the TC. This arrangement promotes the curving of the TC. By Day+2, the anticyclonic wrapping and thinning has resulted in a significantly weaker ridge as compared to a few days prior. For the near-zero curvature cases ( Fig. \ref{fig:lagcurv}f--g), the initial movement of the TC is also directly poleward towards the extratropical ridge. However, in this case, the ridge remains poleward of the TC. There is also significantly less anticyclonic wrapping of ridge. The TC-ridge system takes the form of a cyclonic-anticyclonic vortex pair similar to the rapid tangential deceleration composite. 

The tracks in Fig. \ref{fig:curv_track} clearly show the the propagation of the system. For rapid curvature acceleration (Fig. \ref{fig:curv_track}a), the TC track shows a recurving typhoon. The track of the ridge confirms the initial eastward motion, followed by a poleward shift after parts of it wrapped around the TC and thinned. By Day+2, we do not observe a merger of the tracks that happens in the case of rapid tangential acceleration (Fig. \ref{fig:tang_track}a). The tracks for near-zero curvature acceleration ( (Fig. \ref{fig:curv_track}a) are somewhat similar to the rapid tangential deceleration (Fig. \ref{fig:tang_track}b). The key point here is that, although the TC moves poleward, TC-ridge system is nearly stationary in the zonal direction. 


Overall, the ensemble average composites show that when we separate TCs on the basis of their acceleration, we get two broad sets of synoptic scale patterns. In the case of rapid tangential acceleration, we recover the arrangement of the TC and a developing wavepacket that resembles the canonical ET process. For rapid deceleration and extremes of curvature acceleration, we get a starkly different pattern that takes the form of a vortex dipole. We return to this point later in the discussion section.

%-------------------------------------------------------
\section{Extratropical Transition}
In the previous section, we showed that the composite synoptic-scale flow associated with rapidly accelerating TCs resembles a pattern that is favorable for ET.  However, this does not  imply that all storms in the composite underwent ET. Some TCs may begin the process of ET but dissipate before its completion \citep[e.g.,][]{KRT2010A}. We now consider TC motion from a different perspective by considering only those storms that completed the transformation from being tropical to extratropical.  \cite{HE2006} found that the time taken for ET completion can vary considerably. For the storms that they examined, this ranged from 12--168 hours. To get a sense of the temporal evolution relative to ET completion, we examine composite TC speed and acceleration as a function of time. For this, we only considered those Atlantic storms during 1966--2019 that were classified as \emph{tropical} at some time and subsequently underwent ET. Of the 689 candidate storms that passed the three-day threshold, 18 storms were never classified as \emph{tropical}. Of the remaining 671 storms, 274 were classified as \emph{tropical} initially and ET at a later time. This yields a climatological ET fraction of 41\%. It is slightly lower than the fraction of 44\% over 1979--2017 in \cite{BCSEH2019a}, and 46\% over 1950-1993 in  \cite{HE2001}.  The mean and median latitude of ET completion in our data set were, respectively, 40.5$^\circ$N and 41.5$^\circ$ N. This is consistent with \cite{HE2001} who found that the highest frequency of ET in the Atlantic occurs between the latitudes of 35$^o$N--45$^o$N.



Fig. \ref{fig:ET_SA} shows the composite accelerations and speed relative to the time of ET. Hour 0 was defined as the first instance in the IBTrACS where the storm nature was designated as ET. We interpret this as the time of ET completion. This graph shows that forward speed of transitioning tropical storms reaches a peak around the time of ET completion. The tangential acceleration peaks about 18 hours prior to that. The curvature acceleration appears to steadily increase up to the time of ET and stabilizes thereafter. The point here is that that peak  tangential acceleration of TCs precedes ET completion. The rapid increase in the speed prior to ET completion time is a direct outcome of the interaction with the extratropical baroclinic wavepacket. 


\section{Trends}

Figure \ref{fig:tang_cdf} compares the cumulative probability distribution (CDF) of tangential accelerations taken from all months over two 10-year periods: 1966--1975 and 2010--2019. Three regions are shown: Entire Atlantic,  0--30$^o$N, and 20--40$^o$N. The CDFs show that the median value of tangential acceleration does not change much between the three decades. Notably, however, there is a clear reduction in the frequency of rapid acceleration and deceleration. Figure \ref{fig:curv_cdf} shows CDFs for curvature acceleration. Here too, there is a similar shift towards less frequent rapid acceleration in the past decade compared to 1966--1975. There is a small compensating increase in frequency of relatively less rapid acceleration values. In the case of  0--20$^o$N, the increase is seen for values below the 90$^\text{th}$\% while for 20--40$^o$N, it is seen for values below the median.


As noted earlier from Fig. \ref{fig:distNAMonth}, the magnitude of accelerations increases steadily going from July--November. Fig

CDFs for individual months to get a sense of how the 


In the preceding sections, we showed that rapid acceleration or deceleration of a TC is typically associated with an interaction with extratropical stormtrack. The synoptic-scale pattern associated with these two extremes are distinct in the phasing of the TC and the extratropical wavepacket. It is of interest to see if the differences of the CDFs (Figure \ref{fig:tang_cdf}) are related to long-term trends in TC acceleration. The motivation being that it can inform us about potential trends in the nature of TC-baroclinic stormtrack interaction. For this, we consider rapid acceleration and deceleration separately. We do this for two reasons. First, as seen in Fig.  \ref{fig:tang_cdf}, the median acceleration does not change significantly and this is also the case for the mean acceleration (not shown). Second, the synoptic patterns associated with rapid acceleration and rapid deceleration of TCs are entirely different. We consider three threshold values to calculate trends. These are 90$^\text{th}$\%, 80$^\text{th}$\% and 60$^\text{th}$\% values of both categories. The trends over 1966--2019 are calculated using the annual-average acceleration exceeding these threhsolds for for Atlantic TCs after excluding instances of NR and ET. The trends and the associated P-value calculated using two different methods - (a) linear regression, and (b) the Thiel-Sen linear estimate with the Mann-Kendal significance.

Table \ref{tab:atl_accelTrend} presents the trends for the entire basin and individual latitude bands. The key result here is that the annual mean tangential acceleration of TCs over the Atlantic shows a robust decreasing trend over the period 1966-2019. This is true for full Atlantic basin as well as for individual latitude bands. In particular, the negative trend is strongest for the P$_{90}$ and P$_{80}$thresholds within the latitude range of 20--40$^o$N. This is where which TCs are most likely to interact with extartropical wavepackets and experience ET \citep[e.g.][]{HE2001}. The P$_{60}$ threshold is mostly consistent although the trends increase with latitude. As noted earlier, this threshold being close to the median value, reflects the fact that TCs are nearly equally likely to accelerate and decelerate at lower latitudes and slighly more likely to accelerate at higher latitudes. 



%------------------------------------------------------------
Table \ref{tab:trend_speed} shows trends for annual-average speed of TCs. In this table, we also include 1949--2019 and 1949--2016 in order to compare with Kossin (2018). For the same reason, we also show the trend for all storms in the IBTraCS that include instances that were classified as ET and NR as well as the trend for western North Pacific. Kossin (2018) reported a trend of -0.02 km hr$^{-1}$ year$^{-1}$ for the entire Atlantic basin over 1949--2016. Our results of -0.019 and -0.021 for the linear regression and Thiel-Sen estimate, respectively, are consistent with that. However, when we examine the trends for the same period, but after excluding ET and NR track locations, the trends become negligible. On the other hand, for the entire western North Pacific over 1949--2016, we note significant trends of  -0.025 and -0.030 km hr$^{-1}$ year$^{-1}$.  This is consistent, albeit slightly lower than what was reported in  Kossin (2018). However, the trends over the Satellite era of 1966--2019 are not statistically significant. This is also true over the same period for the latitude band 20--40$^o$N 

Focusing on the Satellite era of 1966-2019,  Figure \ref{fig:trend_speed}  shows the annual-averaged TC speed for all storms (purple) and storms exclusing ET and NR instances (green). While the two timeseries show some trends, neither are statistically significant (Table \ref{tab:trend_speed}).  In fact, none of the trends in the different categories for the Atlantic -- whether the full basin or for the latitude band 20--40$^o$N -- are statistically significant. Indeed, the trend reported by Kossin (2018) for all storms in the Atlantic basin was also not statistically significant. 

Figure \ref{fig:ts_speed} shows the timeseries of annual mean forward speed of tropical cyclones within the entire Atlantic basin. Two curves are shown. The purple curve includes all track points for storms in the IBTRaCS dataset with lifetime exceeding 3 days while the green curve excludes instances when storms were classified as ET or NR. Table \ref{tab:atl_sp} shows the trend and the associated P-value calculated using two different methods - (a) linear regression, and (b) the Thiel-Sen linear estimate with the Mann-Kendal significance. We note that both curves show interannual variation (Fig. \ref{fig:ts_speed}) but the trend over the period 1966--2019 in neither curve is staistically significan (Table \ref{tab:atl_sp}). For comparison, we also include in this table, the trends for all Western North Pacific storms. Here too there is no significant trend over 1966--2019. However, over 1949--2019, there is a significant slowing of TCs in this basin. We also include in the table, the trends for the latitude band 20--40$^o$N over which TCs are most likely to interact with extartropical wavepackets and experience ET \citep[e.g.][]{HE2001}. For this again, there is no significant trend in the Atlantic for either periods. A relatively more robust slowdown of TC speed is seen in the Western North Pacific over  1949--2019, with the Mann-Kendall test showing a P-value of 0.05.

%------------------------------------------------------------



\section{Discussion}
It is tempting to draw an analogy with a dipole block \citep[e.g.,][]{MCW1980,PH2003}, which is an important mode of persistent anomalies in the atmosphere. The canonical dipole block is commonly described as a vortex pair comprised of a warm anticyclone and a low-latitude cut-off cyclone \citep[e.g.,][]{MBDAGE2006}. 

Energy dispersion in both sets of tang accel. comps.

 This entails poleward motion of the TC and interaction with the extratropical baroclinic stormtrack.

Others may decay even earlier owing to detrimental large scale shear or cooler ocean heat content. 


From the preceding sections, it is evident that rapid changes in speed are mediated by interactions with extratropical wave

Begin by noting that the rapid accel indicates a type of interaction that looks like one that favors ET irrespective of whether ET happens or not. This needs to be in the abstract as well as conclusions.

account for:

1) During the peak Atlantic hurricane season, the magnitude of the mean tangential and curvature accelerations increase steadily from July--November. 


\conclusions  %% \conclusions[modified heading if necessary]
TEXT

%% The following commands are for the statements about the availability of data sets and/or software code corresponding to the manuscript.
%% It is strongly recommended to make use of these sections in case data sets and/or software code have been part of your research the article is based on.

%\codeavailability{TEXT} %% use this section when having only software code available

%\dataavailability{TEXT} %% use this section when having only data sets available


%\codedataavailability{TEXT} %% use this section when having data sets and software code available


%\sampleavailability{TEXT} %% use this section when having geoscientific samples available


%\videosupplement{TEXT} %% use this section when having video supplements available


\appendix
\section{}    %% Appendix A



%\subsection{}     %% Appendix A1, A2, etc.


\noappendix       %% use this to mark the end of the appendix section. Otherwise the figures might be numbered incorrectly (e.g. 10 instead of 1).

%% Regarding figures and tables in appendices, the following two options are possible depending on your general handling of figures and tables in the manuscript environment:

%% Option 1: If you sorted all figures and tables into the sections of the text, please also sort the appendix figures and appendix tables into the respective appendix sections.
%% They will be correctly named automatically.

%% Option 2: If you put all figures after the reference list, please insert appendix tables and figures after the normal tables and figures.
%% To rename them correctly to A1, A2, etc., please add the following commands in front of them:



%% Please add \clearpage between each table and/or figure. Further guidelines on figures and tables can be found below.

\authorcontribution{TEXT} %% this section is mandatory

\competinginterests{TEXT} %% this section is mandatory even if you declare that no competing interests are present

\disclaimer{TEXT} %% optional section

\begin{acknowledgements}
This work was supported by NSF award 
\end{acknowledgements}


%% REFERENCES
\bibliographystyle{copernicus}
\bibliography{aiyyer_references.bib}
%%
%% URLs and DOIs can be entered in your BibTeX file as:
%%
%% URL = {http://www.xyz.org/~jones/idx_g.htm}
%% DOI = {10.5194/xyz}


%% LITERATURE CITATIONS
%%
%% command                        & example result
%% \citet{jones90}|               & Jones et al. (1990)
%% \citep{jones90}|               & (Jones et al., 1990)
%% \citep{jones90,jones93}|       & (Jones et al., 1990, 1993)
%% \citep[p.~32]{jones90}|        & (Jones et al., 1990, p.~32)
%% \citep[e.g.,][]{jones90}|      & (e.g., Jones et al., 1990)
%% \citep[e.g.,][p.~32]{jones90}| & (e.g., Jones et al., 1990, p.~32)
%% \citeauthor{jones90}|          & Jones et al.
%% \citeyear{jones90}|            & 1990



%% FIGURES

%% When figures and tables are placed at the end of the MS (article in one-column style), please add \clearpage
%% between bibliography and first table and/or figure as well as between each table and/or figure.

% The figure files should be labelled correctly with Arabic numerals (e.g. fig01.jpg, fig02.png).


\clearpage

%================================================================================================================
%================================================================================================================
%All Tables
%================================================================================================================


\begin{table*}[t]
  \caption{Speed (km hr$^{-1}$), tangential and curvature acceleration (km hr$^{-1}$ day$^{-1}$ ) of Atlantic TCs
   in the IBTRaCs database as a function of latitude. Storm instances classified as ET or NR were excluded.
   N refers to number of 3-hourly track positions in each latitude-bin over the period 1966--2019. }
\begin{tabular}{|cc|ccc|ccc|ccc|}
\hline
& & \multicolumn{3}{|c|}{Speed} & \multicolumn{3}{c|}{Tang. Accel} & \multicolumn{3}{c|}{Curv. Accel}\\
\hline
Latitude        & N     & Mean   & Median & Std Dev. & Mean & Median &  Std Dev.& Mean & Median & Std Dev.\\
Full Basin      & 38822 &  20.85 & 18.87 &  12.2 &   2.27 &  0.68 &  19.5 &  16.42 & 10.88 &  19.0 \\
	  5--15 &  5870 & 23.53 &  23.4 &   9.4 &  0.18 &   0.0 &  14.7 & 11.65 &   8.2 &  13.1 \\
	 10--20 & 13087 & 21.10 &  20.6 &   9.3 & -0.13 &  -0.0 &  15.3 & 12.29 &   8.5 &  13.3 \\
	 15--25 & 13656 & 18.66 &  18.1 &   8.6 &  0.26 &  -0.0 &  16.0 & 13.90 &   9.4 &  15.6 \\
	 20--30 & 13074 & 17.21 &  16.4 &   8.6 &  1.40 &   0.4 &  17.1 & 16.48 &  11.2 &  19.0 \\
	 25--35 & 13905 & 17.74 &  16.2 &  10.2 &  2.74 &   1.3 &  19.7 & 18.04 &  12.2 &  20.6 \\
	 30--40 & 10815 & 21.37 &  18.6 &  13.5 &  5.33 &   3.0 &  22.8 & 19.34 &  13.4 &  21.0 \\
	 35--45 &  5272 & 29.41 &  26.7 &  18.0 &  8.27 &   4.8 &  27.4 & 22.41 &  15.8 &  22.9 \\
	 40--50 &  1607 & 42.37 &  41.0 &  20.8 &  8.75 &   6.4 &  34.9 & 28.45 &  19.9 &  29.0 \\
	 45--55 &   329 & 55.69 &  53.9 &  19.4 &  6.81 &   6.3 &  37.8 & 36.50 &  25.9 &  38.9 \\
\bottomhline
\end{tabular}
%\belowtable{} % Table Footnotes
\label{tab:distNA}
\end{table*}
%==================================================================================


\begin{table*}[t]
  \caption{Trends in Speed (km hr$^{-1}$ year$^{-1}$). \emph{All storms} refers to all instances of a
    system recorded in the IBTraCs. ET refers to storm nature designated as extratropical, while NR refers to instances
    when the storm nature was not recorded. }

\begin{tabular}{|c|cccc|cccc|cccc|}
\hline
& \multicolumn{4}{|c|}{1966--2019}& \multicolumn{4}{|c|}{1949--2019} &  \multicolumn{4}{|c|}{1949--2016}\\
& \multicolumn{2}{c}{LR} & \multicolumn{2}{c|}{MK-TS}& \multicolumn{2}{c}{LR} & \multicolumn{2}{c|}{MK-TS} & \multicolumn{2}{c}{LR} & \multicolumn{2}{c|}{MK-TS}\\
& Trend & P-value & Trend & P-value & Trend & P-value & Trend & P-value & Trend & P-value & Trend & P-value\\
\hline

&  \multicolumn{11}{c}{\bf{Atlantic (Excluding ET,NR)}}  &   \\
Full basin    & -0.007 & 0.70 & -0.008 & 0.62 & -0.004 & 0.77 & -0.007 & 0.48 & -0.002 & 0.90 & -0.006 & 0.63\\
 20--40$^o$N   &  0.005 & 0.82 &  0.005 & 0.83 & -0.010 & 0.49 & -0.017 & 0.16 & -0.007 & 0.66 & -0.015 & 0.25\\
\hline
\hline
&  \multicolumn{11}{c}{\bf{Atlantic (All storms)}}  &   \\
Full basin    & 0.029 & 0.18 &  0.027 & 0.16  & -0.016 & 0.27 & -0.016 & 0.31 & -0.019 & 0.24 & -0.021 & 0.25\\
20--40$^o$N   & 0.016 & 0.50 &  0.015 & 0.37  & -0.010 & 0.50 & -0.014 & 0.30 & -0.008 & 0.61 & -0.014 & 0.38\\

\hline
&  \multicolumn{11}{c}{\bf{Western North Pacific (All storms)}}  &   \\
Full basin    &  0.007 & 0.56 & -0.001 & 0.94 & -0.022 & 0.01 & -0.025 & < 0.01  & -0.025 & 0.01 & -0.030 & <0.01 \\ 
20--40$^o$N   &  0.004 & 0.87 &  0.000 & 0.99 & -0.032 & 0.06 & -0.029 & 0.05 & -0.033 & 0.07 & -0.033 & 0.09\\
\hline
\end{tabular}
%\belowtable{} % Table Footnotes
\label{tab:trend_speed}
\end{table*}

%==================================================================================
%==================================================================================


\begin{table*}[t]
  \caption{Trends in rapid tangential acceleration (km hr$^{-1}$ day$^{-1}$ year$^{-1}$) of Atlantic tropical cyclones over 1966--2019. Storm instances classified as ET or NR were excluded. Three cut-offs for defining rapid are used: Values exceeding 90th, 85th and 60th percentile of all acceleration. }

\begin{tabular}{|c|cccc|cccc|cccc|}
\hline
%& \multicolumn{4}{|c|}{1966--2019}  & \multicolumn{4}{|c|}{1966--2019}\\
& \multicolumn{4}{|c|}{ 90th percentile } & \multicolumn{4}{|c|}{80th percentile }& \multicolumn{4}{|c|}{60th percentile}\\
\bf{Atlantic}  & \multicolumn{2}{c}{LR} & \multicolumn{2}{c|}{MK-TS}& \multicolumn{2}{c}{LR} & \multicolumn{2}{c|}{MK-TS} & \multicolumn{2}{c}{LR} &  \multicolumn{2}{c|}{MK-TS}\\
\bf{(Excluding ET,NR)} & Trend & P-value & Trend & P-value & Trend & P-value & Trend & P-value & Trend & P-value & Trend & P-value\\
\hline
Full basin  & -0.097 & 0.01 & -0.091 & 0.01 & -0.088 &  <0.01 & -0.084 &  <0.01 & -0.051 & 0.01 & -0.050 & 0.02 \\
	  5--25$^o$N & -0.094 & 0.00 & -0.102 & <0.01   & -0.081 & <0.01 & -0.080 & <0.01 & -0.046 & 0.01 & -0.047 & 0.02\\
	 10--30$^o$N & -0.080 & 0.01 & -0.082 & <0.01   & -0.079 & <0.01 & -0.072 & <0.01 & -0.048 & 0.01 & -0.049 & 0.01\\
	 15--35$^o$N & -0.086 & 0.02 & -0.079 & 0.01   & -0.084 & <0.01 & -0.079 & <0.01 & -0.043 & 0.03 & -0.045 & 0.02\\
	 20--40$^o$N & -0.113 & 0.01 & -0.110 & 0.01   & -0.088 & 0.01 & -0.086 & 0.02 & -0.046 & 0.06 & -0.051 & 0.06\\
	 25--45$^o$N & -0.099 & 0.08 & -0.090 & 0.05   & -0.075 & 0.10 & -0.065 & 0.11 & -0.061 & 0.07 & -0.047 & 0.10\\
	 30--50$^o$N & -0.124 & 0.06 & -0.119 & 0.04   & -0.097 & 0.08 & -0.099 & 0.11 & -0.078 & 0.08 & -0.061 & 0.18\\
\hline
\hline
\end{tabular}
%\belowtable{} % Table Footnotes
\label{tab:atl_accelTrend}
\end{table*}




%==================================================================================
%==================================================================================


\begin{table*}[t]
  \caption{Trends in rapid tangential deceleration (km hr$^{-1}$ day$^{-1}$ year$^{-1}$) of Atlantic tropical cyclones. All storm instance, including classified as NR and ET are considered here. Three cut-offs for defining rapid are used: Values exceeeding 90th, 85th and 60th percentile of all acceleration. }

\begin{tabular}{|c|cccc|cccc|cccc|}
\hline
%& \multicolumn{4}{|c|}{1966--2019}  & \multicolumn{4}{|c|}{1966--2019}\\
& \multicolumn{4}{|c|}{90th percentile} & \multicolumn{4}{|c|}{80th percentile }& \multicolumn{4}{|c|}{60th percentile}\\
Atlantic  & \multicolumn{2}{c}{LR} & \multicolumn{2}{c|}{MK-TS}& \multicolumn{2}{c}{LR} & \multicolumn{2}{c|}{MK-TS} & \multicolumn{2}{c}{LR} &  \multicolumn{2}{c|}{MK-TS}\\
(Excluding ET,NR) & Trend & P-value & Trend & P-value & Trend & P-value & Trend & P-value & Trend & P-value & Trend & P-value\\
\hline
Full basin &  0.078 & 0.01 &  0.075 & 0.02&  0.070 & 0.00 &  0.069 & 0.00 &  0.040 & 0.00 &  0.041 & 0.00\\
	  5--25 &  0.103 & 0.00 &  0.108 & 0.00 &  0.086 & 0.00 &  0.093 & 0.00 &  0.067 & 0.00 &  0.073 & 0.00\\
	 10--30 &  0.087 & 0.01 &  0.082 & 0.00 &  0.078 & 0.00 &  0.080 & 0.00 &  0.052 & 0.00 &  0.049 & 0.00\\
	 15--35 &  0.077 & 0.02 &  0.071 & 0.02 & 0.067  & 0.00 &  0.066 & 0.01 &  0.032 & 0.03 &  0.027 & 0.04\\
	 20--40 &  0.064 & 0.03 &  0.068 & 0.04 & 0.054  & 0.01 &  0.051 & 0.02 &  0.014 & 0.29 &  0.015 & 0.31\\
	 25--45 &  0.054 & 0.23 &  0.052 & 0.22 &  0.050 & 0.10 &  0.040 & 0.13 &  0.007 & 0.70 & -0.002 & 0.92\\
	 30--50 &  0.081 & 0.20 &  0.062 & 0.23 &  0.044 & 0.26 &  0.022 & 0.52 &  0.002 & 0.91 &  0.003 & 0.95\\
\hline
\hline
\end{tabular}
%\belowtable{} % Table Footnotes
\label{tab:ATLAC}
\end{table*}


%================================================================================================================
%All Figures
%================================================================================================================



%========================================================
%Figure 1
%========================================================
\clearpage


\begin{figure*}[t]
\includegraphics[height=6cm]{track_map_katrina.png}
\includegraphics[height=6cm]{track_map_irma.png}
\caption{Illustration of the circle-fit and radius of curvature calculations at
five selected locations along the track of hurricanes Katrina (2005) and Irma (2017).}\label{fig:track}
\end{figure*}
%========================================================


\clearpage


%========================================================
%Figure 1
%========================================================

\begin{figure*}[t]
  \centering
    \includegraphics[width=.9\textwidth]{distribution_NA.png}
    \caption{Distribution of (a) Speed; (b) Tangential acceleration and (c) Curvature acceleration of
     Atlantic TCs as a function of latitude. Storm instances classified as ET or NR were excluded. Data from 1966--2019 was binned within 10$^o$-wide overlapping
     latitude bins. Statistics shown are: median (horizontal line within the box), mean (dot), and 10th,
     25th, 75th and 90th percentiles.}
  \label{fig:distNA}
\end{figure*}

\clearpage
%========================================================


%========================================================
%Figure 1
%========================================================

\begin{figure*}[t]
  \centering
  \includegraphics[width=.45\textwidth]{tang_mon_climo.png}
  \includegraphics[width=.45\textwidth]{curv_mon_climo.png}
  \caption{Distribution of tangential and curvature accelerations of Atlantic TCs within the latitude band 20--40$^\circ$N
    as a function of month. Numbers on top of the bars show the percentage of track points that occurred during each month.
    Storm instances classified as ET or NR were excluded}
  \label{fig:distNAMonth}
\end{figure*}

\clearpage
%========================================================


\begin{figure*}[t]
  \centering
    \includegraphics[width=.85\textwidth]{composite_tang_40_30_multilevel_NA.png}
  \caption{Storm-relative composite average geopotential heights (thick orange lines) and anomalies (color shaded) for all TCs located in the latitude bin 30-40$^o$N over the Atlantic. The composite fields are shown for three levels -- 300 hPa, 500 hPa and 850 hPa. In each panel, the composite 1000 hPa anomalous geopotential is shown using thin black contours. All anomalies are defined relative to a long-term synoptic climatology. The contour intervals are: 12 dam, 6 dam and 3 dam for the three levels respectively. The shading interval in dam for the 300 hPa anomaly fields is shown in the label-bar. It is half of the value for the other two levels. The left column is for rapid tangential acceleration and the right column is for rapid tangential deceleration.}
  \label{fig:tangComp}
\end{figure*}
%========================================================

%========================================================

\clearpage

\begin{figure*}[t]
  \centering
    \includegraphics[width=.85\textwidth]{composite_tang_40_30_alllags_NA.png}
  \caption{Storm-relative average 500-hPa geopotential heights (thick orange lines) and anomalies (color shaded) for all TCs located in the latitude bin 30-40$^o$N over the Atlantic. The fields are shown for lags Day-2 to Day+2. In each panel, the composite 1000 hPa anomalous geopotential is shown using thin black contours. All anomalies are defined relative to a long-term synoptic climatology. The contour interval is 6 dam and shading interval in dam is shown in the label-bar. The plus symbol shows the location of the composite TC at Day 0 and the hurricane symbol shows the approximate location at each lags. The left column is for rapid tangential acceleration and the right column is for rapid tangential deceleration}
  \label{fig:lagtang}
\end{figure*}
%========================================================

\clearpage
%========================================================
\begin{figure*}[t]
  \centering
    \includegraphics[width=.85\textwidth]{composite_curv_40_30_multilevel_NA.png}
  \caption{As in Fig. \ref{fig:tangComp}, but for rapid and near-zero curvature acceleration}
  \label{fig:curvComp}
\end{figure*}
%========================================================





\clearpage
%========================================================
\begin{figure*}[t]
  \centering
    \includegraphics[width=.85\textwidth]{composite_curv_40_30_alllags_NA.png}
  \caption{As in Fig. \ref{fig:lagtang}, except for rapid (left column) and near-zero (right column) curvature acceleration.}
  \label{fig:lagcurv}
\end{figure*}
%========================================================

\clearpage
%========================================================
\begin{figure*}[t]
  \centering
    \includegraphics[width=.45\textwidth]{tang_rapid_acel_track.png}
     \includegraphics[width=.45\textwidth]{tang_rapid_decel_track.png}
  \caption{}
  \label{fig:tang_track}
\end{figure*}
%========================================================

%========================================================

\clearpage
%========================================================

\begin{figure*}[t]
  \centering
    \includegraphics[width=.45\textwidth]{curv_rapid_acel_track.png}
     \includegraphics[width=.45\textwidth]{curv_low_acel_track.png}
  \caption{}
  \label{fig:curv_track}
\end{figure*}


%========================================================



\begin{figure*}[t]
  \includegraphics[width=12cm]{ET_relative_accel_speed.png}
  \caption{Composite speed and accelerations relative to time of ET. A single pass of 5-point running average was applied to the speed and tangential acceleration curves. Two passes of the same filter were applied to the curvature acceleration.}\label{fig:ET_SA}
\end{figure*}
%========================================================

%========================================================
%Figure 
%========================================================


\clearpage

%========================================================
%Figure 
%========================================================
\begin{figure*}[t]
  \includegraphics[width=14cm]{tangential_cdf.png}
  \caption{CDF}\label{fig:tang_cdf}
\end{figure*}
%========================================================

%========================================================
%Figure 
%========================================================
\begin{figure*}[t]
  \includegraphics[width=14cm]{aug_cdf_atl.png}
  \includegraphics[width=14cm]{sep_cdf_atl.png}
  \includegraphics[width=14cm]{oct_cdf_atl.png}
  \caption{CDF}\label{fig:tang_month_cdf}
\end{figure*}
%========================================================






%Figure 
%========================================================

\begin{figure*}[t]
  \includegraphics[width=14cm]{curv_cdf.png}
  \caption{CDF}\label{fig:curv_cdf}
\end{figure*}
%========================================================



\clearpage

%========================================================
%Figure 
%========================================================

\begin{figure*}[t]
  \includegraphics[width=7cm]{atl_tang_no_etnr.png}
   \includegraphics[width=7cm]{atl_tang_20_40_no_etnr.png}

  \caption{Annual-mean tangential acceleration exceeding thresholds of 60th, 80th and 90th percentile values for (a) Entire Atlantic; and (b) Latitude range 20-40$^o$N. Instances of storms classified as NR and ET were excluded. The linear trend for each timeseries is shown by the straight line.}\label{fig:atl_speed}
\end{figure*}
%========================================================






\begin{figure*}[t]
  \includegraphics[width=14cm]{atl_speed_trend.png}
\caption{Annual-mean speed of storms in the (a) North Atlantic; and (b) western North Pacific, along with the linear trend. The purple curve is for all storms in the IBTRaCS dataset while the green curve excludes instances when
the storm was classified as ET or NR.}\label{fig:trend_speed}
\end{figure*}
%========================================================




%%% TABLES
%%%
%%% The different columns must be seperated with a & command and should
%%% end with \\ to identify the column brake.
%
%%% ONE-COLUMN TABLE
%
%%t
%\begin{table}[t]
%\caption{TEXT}
%\begin{tabular}{column = lcr}
%\tophline
%
%\middlehline
%
%\bottomhline
%\end{tabular}
%\belowtable{} % Table Footnotes
%\end{table}
%
%%% TWO-COLUMN TABLE
%
%%t
%\begin{table*}[t]
%\caption{TEXT}
%\begin{tabular}{column = lcr}
%\tophline
%
%\middlehline
%
%\bottomhline
%\end{tabular}
%\belowtable{} % Table Footnotes
%\end{table*}
%
%%% LANDSCAPE TABLE
%
%%t
%\begin{sidewaystable*}[t]
%\caption{TEXT}
%\begin{tabular}{column = lcr}
%\tophline
%
%\middlehline
%
%\bottomhline
%\end{tabular}
%\belowtable{} % Table Footnotes
%\end{sidewaystable*}
%
%
%%% MATHEMATICAL EXPRESSIONS
%
%%% All papers typeset by Copernicus Publications follow the math typesetting regulations
%%% given by the IUPAC Green Book (IUPAC: Quantities, Units and Symbols in Physical Chemistry,
%%% 2nd Edn., Blackwell Science, available at: http://old.iupac.org/publications/books/gbook/green_book_2ed.pdf, 1993).
%%%
%%% Physical quantities/variables are typeset in italic font (t for time, T for Temperature)
%%% Indices which are not defined are typeset in italic font (x, y, z, a, b, c)
%%% Items/objects which are defined are typeset in roman font (Car A, Car B)
%%% Descriptions/specifications which are defined by itself are typeset in roman font (abs, rel, ref, tot, net, ice)
%%% Abbreviations from 2 letters are typeset in roman font (RH, LAI)
%%% Vectors are identified in bold italic font using \vec{x}
%%% Matrices are identified in bold roman font
%%% Multiplication signs are typeset using the LaTeX commands \times (for vector products, grids, and exponential notations) or \cdot
%%% The character * should not be applied as mutliplication sign
%
%
%%% EQUATIONS
%
%%% Single-row equation
%
%\begin{equation}
%
%\end{equation}
%
%%% Multiline equation
%
%\begin{align}
%& 3 + 5 = 8\\
%& 3 + 5 = 8\\
%& 3 + 5 = 8
%\end{align}
%
%
%%% MATRICES
%
%\begin{matrix}
%x & y & z\\
%x & y & z\\
%x & y & z\\
%\end{matrix}
%
%
%%% ALGORITHM
%
%\begin{algorithm}
%\caption{...}
%\label{a1}
%\begin{algorithmic}
%...
%\end{algorithmic}
%\end{algorithm}
%
%
%%% CHEMICAL FORMULAS AND REACTIONS
%
%%% For formulas embedded in the text, please use \chem{}
%
%%% The reaction environment creates labels including the letter R, i.e. (R1), (R2), etc.
%
%\begin{reaction}
%%% \rightarrow should be used for normal (one-way) chemical reactions
%%% \rightleftharpoons should be used for equilibria
%%% \leftrightarrow should be used for resonance structures
%\end{reaction}
%
%
%%% PHYSICAL UNITS
%%%
%%% Please use \unit{} and apply the exponential notation


\clearpage
\appendixfigures  %% needs to be added in front of appendix figures


%\begin{figure}[ht]
%  \centering
%    \includegraphics[width=.7\textwidth]{distributionWP.png}
%  \caption{}
%  \label{fig:distWP}
%\end{figure}


\appendixtables   %% needs to be added in front of appendix tables


\begin{table*}[t]
\caption{Speed (km hr$^{-1}$), tangential and curvature acceleration  (km hr$^{-1}$ day$^{-1}$ ) of all western North Pacific storms as a function of latitude: N refers to number of 3-hourly track positions in each latitude-bin over the period 1966--2019. Storm positions corresponding to 
nature labels "ET" and "NR" were included.}

\begin{tabular}{cc|ccc|ccc|ccc}
\tophline
& & \multicolumn{3}{|c|}{Speed} & \multicolumn{3}{c|}{Tang. Accel} & \multicolumn{3}{c}{Curv. Accel}\\
\hline
       Latitude &     N &  Mean & Median & Std Dev. & Mean & Median &  Std Dev.& Mean & Median & Std Dev.\\
 	  5--15 & 40020 & 17.96 &  17.3 &   8.5 & 0.29 &   0.1 &  16.7 & 14.79 &   9.1 &  19.6 \\
	 10--20 & 53917 & 17.22 &  16.5 &   8.1 &  0.12 &   0.0 &  16.6 & 15.22 &   9.7 &  19.4 \\
	 15--25 & 49678 & 17.02 &  16.1 &   8.6 &  1.13 &   0.6 &  17.9 & 16.23 &  10.4 &  19.9 \\
	 20--30 & 33365 & 19.15 &  17.4 &  10.9 &  3.07 &   1.9 &  20.9 & 18.11 &  11.9 &  20.9 \\
	 25--35 & 21128 & 24.18 &  20.6 &  15.4 &  5.61 &   3.4 &  26.7 & 20.88 &  14.3 &  22.7 \\
	 30--40 & 13286 & 31.90 &  27.4 &  20.2 &  7.01 &   4.4 &  35.0 & 26.39 &  18.2 &  28.3 \\
	 35--45 &  8310 & 40.20 &  36.5 &  22.6 &  2.89 &   2.4 &  44.6 & 34.83 &  23.7 &  37.2 \\
	 40--50 &  5741 & 41.00 &  37.7 &  22.7 & -3.44 &  -1.0 &  48.3 & 40.08 &  27.0 &  42.9 \\
	 45--55 &  3516 & 37.32 &  33.7 &  21.2 & -5.78 &  -2.3 &  47.4 & 40.16 &  26.7 &  44.3 \\
\hline
\end{tabular}
%\belowtable{} % Table Footnotes
\label{tab:distWP}
\end{table*}


\clearpage

\begin{figure*}[t]
  \centering
    \includegraphics[width=.9\textwidth]{distribution_WP.png}
    \caption{Distribution of (a) Speed; (b) Tangential acceleration and (c) Curvature acceleration of all
     western North Pacific storms as a function of latitude. Storm instances classified as ET or NR were included. Data from 1966--2019 was binned within 10$^o$-wide overlapping
     latitude bins. Statistics shown are: median (horizontal line within the box), mean (dot), and 10th,
     25th, 75th and 90th percentiles.}
  \label{fig:distWP}
\end{figure*}

\end{document}
