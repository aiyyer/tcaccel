%% Copernicus Publications Manuscript Preparation Template for LaTeX Submissions
%% ---------------------------------
%% This template should be used for copernicus.cls
%% The class file and some style files are bundled in the Copernicus Latex Package, which can be downloaded from the different journal webpages.
%% For further assistance please contact Copernicus Publications at: production@copernicus.org
%% https://publications.copernicus.org/for_authors/manuscript_preparation.html


%% Please use the following documentclass and journal abbreviations for preprints and final revised papers.

%% 2-column papers and preprints
\documentclass[wcd,manuscript]{copernicus}
% Weather and Climate Dynamics (wcd)


%% \usepackage commands included in the copernicus.cls:
%\usepackage[english]{babel}
%\usepackage{tabularx}
%\usepackage{cancel}
%\usepackage{multirow}
%\usepackage{supertabular}
%\usepackage{algorithmic}
%\usepackage{algorithm}
%\usepackage{amsthm}
%\usepackage{float}
%\usepackage{subfig}
%\usepackage{rotating}


\begin{document}

\title{Along-Track Acceleration of Tropical Cyclones: A Proxy For Extratropical Interactions}


% \Author[1]{given_name}{surname}

\Author[1,*]{Anantha}{Aiyyer}
\Author[1,*]{Terrell}{Wade}

\affil[1]{Department of Marine, Earth and Atmospheric Sciences, North Carolina State University}


%% The [] brackets identify the author with the corresponding affiliation. 1, 2, 3, etc. should be inserted.

%% If an author is deceased, please mark the respective author name(s) with a dagger, e.g. "\Author[2,$\dag$]{Anton}{Aman}", and add a further "\affil[$\dag$]{deceased, 1 July 2019}".

%% If authors contributed equally, please mark the respective author names with an asterisk, e.g. "\Author[2,*]{Anton}{Aman}" and "\Author[3,*]{Bradley}{Bman}" and add a further affiliation: "\affil[*]{These authors contributed equally to this work.}".


\correspondence{A. Aiyyer (aaiyyer@ncsu.edu)}

\runningtitle{Tropical Cyclone Acceleration}

\runningauthor{Aiyyer and Wade}



\received{}
\pubdiscuss{} %% only important for two-stage journals
\revised{}
\accepted{}
\published{}

%% These dates will be inserted by Copernicus Publications during the typesetting process.


\firstpage{1}

\maketitle


\begin{abstract}
  It is well known that rapid changes in tropical cyclone (TC) motion occur during interaction with extratropical wavetrains. While TC speed of motion, in general, has received much attention in the past, acceleration has not. Here, using a large data sample, we formally examine the synoptic-scale patterns associated with  tangential and normal components of TC acceleration. During rapid acceleration, a developing wavepacket in the extratropics is present and the TC ?????? with the TC straddled by an upstream trough and downstream ridge. This pattern is consistent with the process of extratropical transition (ET) of TCs. In contrast, during near-zero or rapid deceleration, the extratropical wavepacket remains poleward, with a ridge  located directly north of the storm center.  On average, the tangential acceleration peaks about 18 hours prior to the ET designation. The curvature acceleration increases before ET and stabilizes thereafter.
%
 We also find a robust decrease in the annually averaged values of rapid acceleration over the past 38 years within the latitude band most commonly associated with ET over the Atlantic. Consequently, we hypothesize that trends in rapid acceleration can be used as an independent metric to assess changes in the extratropical stormtrack activity that mediate ET.
\end{abstract}


%\copyrightstatement{TEXT}


\introduction
The motion of a tropical cyclone (TC) is known to be  governed by the large-scale environment surrounding it \citep[e.g.][]{Hodanish1993}. As noted by \cite{E2018}, it was recognized early on that the forward speed of a TC can be closely approximated by the surrounding wind field.  It is also understood that a TC is not an isolated vortex that is passively carried by a current. The background environment, typically, also has attendant gradients of potential vorticity, moisture, and wind shear. As also summarized by \cite{E2018}, a TC actively responds to these external stimuli. As a consequence, not only its structure and intensity are impacted, the large-scale flow may also change significantly. This, in turn, may also affect its motion. An example of the response to background gradient of potential vorticity is the generation of the so called $\beta-gyres$, which contribute to TC motion. Similarly, vertical shear can modulate the intensity of the TC via ventilation of low entropy air into the storm core. This in turn can also impact TC motion. On the other hand, a TC can influence the large-scale flow by exciting baroclinic waves in the extratropical stormtrack, leading to rapid downstream development and rearrangement of the flow. Thus, it can be contended that a TC is always interacting with its environment and the interaction is manifested in its motion.

% 

The aspects of TC motion that are particularly important for operational forecasts are track and forward speed. The track garners much attention for obvious reasons -- it  informs  potential locations that will be affected by a storm. TC forward speed is an important parameter that impacts storm surge, intensity changes and precipitation. There is a large body of literature dealing with various research problems related to TC tracks and speed. They are extensively documented in several recent reviews \citep[e.g.,][]{E2018,other}. In contrast, while forecasters have long known that TCs speed up rapidly while approaching an extratropical wave-train, relatively less attention has been devoted in the research literature to details of TC acceleration. 





As noted by \cite{E2018}, steady progress has been made in our understanding of TC motion and this is reflected in our ability to forecast their tracks.




%To our knowledge, relatively few studies have examined the synoptic patterns associated with various thresholds of TC acceleration. 


\section{Data}
We used the IBTrACS v4 \citep{KnappIBTRaCS} data for track information. We did not apply a threshold for maximum sustained winds, but discarded all storms that lasted less than two days.  We focused on the years 1966--2019, spanning the \emph{satellite era} for the North Atlantic and Pacific basins (e.g.. Beven et al. 2003; Landea 2007; Vecchi and Knutson), for which the TC data is deemed relatively more reliable as compared to previous years. Selected atmospheric fields were taken from the the European Center for Medium Range Weather Forecasting's (ECMWF) ERA-Interim (ERAi) reanalysis \citep{DUS11} for the period 1981--2016.


 As our intent here was to examine the motion of tropical storms, in subsequent sections unless explicitly stated,  we focused on the track data prior to the completion of ET. To ascertain whether a tropical storm underwent ET, we used the \emph{nature} designation in the IBTraCS database that relies on the judgement of the forecasters from one or more agencies responsible for the ocean basin.  Admittedly, there will be some subjectivity in this designation. An alternative would be to employ a metric such as the cyclone phase space \citep{Hart2003}, usually calculated using meteorological fields from global numerical weather prediction models (e.g., reanalysis data). As we wish to be independent of modeled products to characterize the storm, we choose to use the forecaster designated storm nature. Furthermore, \cite{BCSEH2019a} found that the phase space calculation was sensitive to the choice of the reanalysis model. Occasionally, the nature of the storm is not recorded (NR) or, if there is an inconsistency among the agencies, it is designated as mixed (MX). In the north Atlantic, only a small fraction of track data ($\approx 0.5\%$) in the IBTrACs is designated as NR or MX. However, the fraction is higher in the North Pacific ($\approx 14\%$), and we return to this point in a subsequent section where it could introduce some uncertainty in our conclusion for this basin.
 
 
\section{Tangential and curvature acceleration}
The acceleration of a hypothetical fluid element moving with the center of a TC can be written as:
%
\begin{equation}
    \mathbf{a} = \frac{dV}{dt} \mathbf{\hat{s}} + \frac{V^2}{R} \mathbf{\hat{n}} 
    \label{eq:acc}
\end{equation}
%
where $V$ is the forward speed and $R$ is the radius of curvature of the track at a given location. Here, $\mathbf{\hat{s}}$ and $\mathbf{\hat{n}}$ are orthogonal unit vectors in the so-called \emph{natural coordinate} system. The former is directed along the TC motion. The latter is directed along the radius of curvature of the track.  The first term in equation \ref{eq:acc} is the tangential acceleration and the second is the curvature or normal acceleration. The speed $V_j$ at any track location (given by index j)  is calculated as:
%
\begin{equation}
V_j = \frac{(D_{j,j-1} +  D_{j,j+1})}{\delta t}
\end{equation}
%
where $D$ refers to the distance between the two consecutive points, indexed as shown above, along the track. Since we used 3-hourly reports from the IBtracs, $\delta t = 6$ h. The tangential acceleration was calculated from the speed using centered differences. To calculate the curvature acceleration, it is necessary to first calculate the radius of curvature, R. For this, we  fitted a circle at each track point using the method described in Appendix A. Figure \ref{fig:track} illustrates R at five selected locations along the the track of hurricane Katrina (2005). 

%approximated as follows. The latitude-longitude positions of the TC track are projected to a 2-dimensional surface using the Mercator transformation. Then, a circle is fitted to each set of three consecutive track points denoted by $\mathbf{X}_{j-1}, \mathbf{X}_j, \mathbf{X}_{j+1}$. The resulting circle is used to assign the radius of curvature of the track point  $\mathbf{X}_j$.  


\section{Basic Statistics}

During 1966--2019, a total of 688 storms in the North Atlantic and 1676 storms in the western North Pacific passed the lifetime threshold of 2 days. 



Figures \ref{fig:distNA} and \ref{fig:distWP} shows some basic statistics for TC speed and accelerations as a function of latitude. TC locations were sorted in $10^\circ$-wide latitude bins for this. Tables \ref{tab:AC1}-\ref{tab:AT2} provide the same data. The average speed of all North Atlantic storms (including over-land) is about 21 km/hr. The same for the western Pacific storms is about 19 km/hr. As expected, TC speed clearly varies with latitude. It is lower in the subtropics as compared to other latitudes. It increases sharply in the vicinity of $40^{o}$ N. 


The tangential acceleration, as defined in eq. \ref{eq:acc}, can be positive or negative. The mean and median tangential acceleration in both basins remain near-zero equatorward of $30^\circ$N. This suggests that TCs in this region tend to translate steadily, or are equally likely to accelerate and decelerate. Poleward of $30^\circ$N, they are substantially positive, indicating that TCs in these latitudes are subject to rapid acceleration. For example, the mean tangential acceleration in the $30^\circ-40^\circ$N latitude band is about 5.4  km/hr day$^{-1}$ in the Atlantic and nearly 8  km/hr day$^{-1}$ in the Western Pacific.  The curvature acceleration, by our definition, takes only positive values and, and on average, steadily increases with latitude. The distribution of TC speed and tangential acceleration is relatively symmetric about the median value as compared to the curvature acceleration.

%In essence, we draw a distinction between the propagation of extratropical and tropical systems. We reason that an extratropical cyclone embedded in the stormtrack is subject to baroclinic development and dynamics of Rossby wavepackets \citep[e.g.][]{K2019}.  In contrast, a tropical system is primarily subject to steering by the background flow. \cite The importance of this distinction will be emphasized further in a subsequent section.




\section{Ensemble average flow for extremes in acceleration}

We now examine the flow pattern associated with extremes in tangential and curvature acceleration of tropical storms. For this, storm-relative ensemble average fields are constructed as follows:

\begin{itemize}
    \item All TC track locations were binned into 10$^\circ$ wide latitude strips (e.g.,  10-20$^\circ$ N, 20-30$^\circ$ N, 30-40$^\circ$ N). For brevity, only results for the 30-40$^\circ$ N bin are discussed here.
    
    \item Taking all TC positions in each latitude bin, the mean ($\bar{A}$) and standard deviation ($\sigma$) of  acceleration (tangential and curvature separately) were calculated. 

    \item The data in each bin were separated into three categories: (a) Rapid acceleration ($A > \bar A + s \sigma_A$); (b) Near Zero acceleration ($ |A - \bar{A}| < .025\sigma_A$); and (c) Rapid deceleration ($A  < \bar A - s\sigma_A$). We show results for s=1 here and note that they do not change qualitatively when we tried other values of s (=1.25, 1.5). TC positions that did not fall in these categories are discarded. Note: A TC could appear more than once within this latitude range at different times depending on its motion.
    
    \item Using the dates corresponding to each track position, ensemble average of large scale fields (e.g. geopotential) were calculated after shifting the grids such that the center of all storms are coincident. The centroid position of storms in that latitude range was used for the reference center and the reference time is denoted as Day-0.  This yields a storm-relative ensemble average view of the large scale atmospheric fields.
    
    \item  For curvature acceleration, which as defined here takes positive values, only the first two categories (rapid and near-zero) were considered.
    
\end{itemize}

\subsection{Tangential Acceleration}

%------------------------------------------------------------------------------------------
\begin{figure*}[t]
  \centering
    \includegraphics[width=.95\textwidth, trim={0cm 8cm 0cm 4cm},clip]{composite_tang_40_30_NA_zero.pdf}
  \caption{Storm-relative composite average geopotential heights (thick solid lines) and anomalies (color shaded) for all TCs located in the latitude bin 30-40$^o$N over the Atlantic. The composite fields are shown at three levels -- 300 hPa, 500 hPa and 850 hPa. In each panel, the anomalous 1000 hPa geopotential is shown using thinner purple contours. Left, center and right columns show the fields for rapid acceleration, near-zero and rapid deceleration respectively.}
  \label{fig:lag0CompZ}
\end{figure*}


Figure \ref{fig:lag0CompZ} shows the composite geopotential heights (thick solid lines) and their anomalies (color shaded) calculated using all Atlantic TC locations within 30-40$^\circ$ N for the three categories of tangential acceleration. The anomalies were defined relative to a long term (1980--2016) climatology. The composite fields are shown at three levels -- 300 hPa, 500 hPa and 850 hPa. In each panel, the anomalous 1000 hPa geopotential is shown using thinner purple contours. It highlights the composite TC as well as the surface development within the stormtrack. 

The ensemble average at reference time (Day=0) for rapid tangential acceleration cases (left column of Fig. \ref{fig:lag0CompZ}) shows the composite TC interacting with within a well defined extratropical wavepacket. The TC is straddled by an upstream trough and a downstream ridge. At 500 hPa, the geopotential anomalies of the TC and the upstream trough are clearly connected. The 1000 hPa geopotential anomaly field shows the near-surface cyclones and anticyclones of the extratropical stormtrack but is dominated by the cyclonic anomaly associated with the TC. The entire wavepacket shows upshear tilt of geopotential anomalies with height, indicating baroclinic growth. This arrangement of the TC and the extratropical wavepacket is consistent with the synoptic-scale flow that is typically associated with ET \citep[e.g.,][]{KHE2000,MGY2003,RJD2008,RJ2014,K2019}. We note that at this point, all storms in the ensemble average are still classified as \emph{tropical}. Thus, we interpret this composite as pre-ET completion state. 

The ensemble average for near zero tangential acceleration cases (middle column of Fig. \ref{fig:lag0CompZ}) shows a very different synoptic scale pattern. The extratropical wavepacket is substantially poleward, with a ridge immediately north of the composite TC. The geopotential anomalies of the extratropical wavepacket and the composite TC appear to be distinct with no evidence of merger. Interestingly, the ensemble average for rapid tangential deceleration cases (right column of Fig. \ref{fig:lag0CompZ}) appears to be similar to the near-zero acceleration cases, albeit the downstream extratropical ridge is a little stronger in the latter. 

%------------------------------------------------------------------------------------------
% 
To get a sense of the temporal evolution of the TC-extratropical wavepacket system, we now show the composites from Day-1 to Day+1 in Fig. \ref{fig:lagCompZ}. As in the previous figure, the three categories of acceleration are arranged in the respective columns. The rows now show the lagged composites of 500 hPa geopotential height (thick contours) and anomalies (color shaded). In each panel, the corresponding composite 1000 hPa geopotential height anomalies are shown by purple contours. 

The ensemble average for rapid tangential acceleration cases (left column of Fig. \ref{fig:lagCompZ}) shows a TC moving rapidly towards a extratropical wavepacket. At day-1, the TC circulation is relatively symmetric as depicted by the contours of 1000 hPa geopotential height anomalies. At 500 hPa, the upstream trough is still at a distance while the downstream ridge has linked with the TC outflow. By Day+1, the composite storm has moved further poleward and eastward and is now located between the upper-level upstream trough and downstream ridge in a position that is optimal for surface baroclinic development. The 1000 hPa geopotential field is now asymmetric with a characteristic signal of a cold front.  Over the course of three days, at 500 hPa, the upstream trough as well as the downstream ridge-trough couplet appear to amplify while simultaneously propagating downstream. The 500 hPa geopotential gradient poleward of the storm is also amplified, indicating a strengthening jet streak. These developments are generally consistent with the process of ET. As summarized in \citet{K2019}, several mechanisms are at play: The poleward moving TC perturbs the extratropical flow and excites a developing Rossby wavepacket that disperses energy downstream. Furthermore, the outflow of the TC is a source of low potential vorticity (PV) air that further reinforces the downstream ridge. Overall, it is evident that the ensemble average evolution of rapid acceleration cases resembles the canonical ET process.

On the other hand, for the near zero acceleration cases (middle column Fig. \ref{fig:lagCompZ}), the composite storm remains equatorward of the extratropical wavepacket and maintains a nearly symmetric structure. Furthermore, the extratropical wavepacket shows little change in amplitude and the constituent troughs and ridges remain nearly stationary, with little evidence of downstream development.  Thus, the near zero acceleration cases are not likely to be involved in ET. The rapid deceleration composite (right column Fig. \ref{fig:lagCompZ}) also shows that the extratropical wavepacket is poleward of the TC. However, by Day+1, there is evidence of amplification of the downstream ridge-trough couplet indicating development in the extratropical stormtrack related to the interaction with the TC.  This indicates a likelihood of ET at a later time.



\begin{figure*}[t]
  \centering
    \includegraphics[width=.95\textwidth, trim={0cm 8cm 0cm 4cm},clip]{composite_tang_40_30_NA_500hPa.pdf}
  \caption{}
  \label{fig:lagCompZ}
\end{figure*}
%-------------------------------------------------------

\subsection{Tangential Acceleration}



%-------------------------------------------------------
\section{Extratropical Transition}

In this section, we further examine the relationship between TC acceleration and ET. 


Fig. \ref{fig:AS} shows the temporal evolution of TC accelerations and speed relative to the time of ET. This was calculated by identifying all TCs that underwent ET and assigning hour 0 to the first instance of ET designation in the track. The figure shows the ensemble average over all 

\subsection{Trends}


\conclusions  %% \conclusions[modified heading if necessary]
TEXT

%% The following commands are for the statements about the availability of data sets and/or software code corresponding to the manuscript.
%% It is strongly recommended to make use of these sections in case data sets and/or software code have been part of your research the article is based on.

%\codeavailability{TEXT} %% use this section when having only software code available

%\dataavailability{TEXT} %% use this section when having only data sets available


%\codedataavailability{TEXT} %% use this section when having data sets and software code available


%\sampleavailability{TEXT} %% use this section when having geoscientific samples available


%\videosupplement{TEXT} %% use this section when having video supplements available


\appendix
\section{}    %% Appendix A
A standard approach to calculating the radius of curvature, given a set of discrete points along a curve (in our case, a TC track) is to fit a circle through three consecutive points. For a curved line on a sphere, it can be shown that (Note: While the following expressions were derived independently for this work, no claim of originality is made since they are based on elementary principles of geometry):


\begin{subequations}
\renewcommand{\theequation}{\theparentequation.\arabic{equation}}
\begin{equation}
R  = R_e sin^{-1} {\left(\sqrt{\frac{2 d_{12} d_{13} d_{23}} {(d_{12} +d_{13} +d_{23})^2 -2(d_{12}^2 +d_{13}^2 +d_{23}^2)}}\right)}
\end{equation}
%
where $R$ is the radius of curvature, $R_e$ is the radius of the Earth, and the $d$ terms are expressed as follows:
%
\begin{gather}
 d_{12} = 1-(\cos{T_1}\cos{T_2}\cos{(N_2-N_1)} + \sin{T_1}\sin{T_2}) \\
 d_{13} = 1-(\cos{T_1}\cos{T_3}\cos{(N_3-N_1)} + \sin{T_1}\sin{T_3}) \\
 d_{23} = 1-(\cos{T_2}\cos{T_3}\cos{(N_3-N_2)} + \sin{T_2}\sin{T_3})
\end{gather}
\end{subequations}
%
where, $T_1$, $T_2$, and $T_3$ are the latitudes of the 3 points while $N_1$, $N_2$, and $N_3$ are the longitudes. The center of the circle is given by the coordinates:
%
\begin{subequations}
\renewcommand{\theequation}{\theparentequation.\arabic{equation}}
\begin{gather}
\tan{\mathbf{T}} = \pm {\frac{\cos{T_1}\cos{T_2}\sin{(N_2-N_1)} + \cos{T_1}\cos{T_3}\sin{(N_1-N_3)} + \cos{T_2}\cos{T_3}\sin{(N_3-N_2)}} {\sqrt{\alpha^2+\beta^2}}} \\ \tan{\mathbf{N}} = -{\frac{\alpha}{\beta}}
\end{gather}
%
Where $\mathbf{T}$ and $\mathbf{N}$ are the latitude and longitude of the circle, respectively and $\alpha$ and $\beta$ are obtained using:
\begin{gather}
\alpha = \cos{T_1}(\sin{T_2}-\sin{T_3})\cos{N_1} + \cos{T_2}(\sin{T_3} -\sin{T_1})\cos{N_2} +\cos{T_3}(\sin{T_1}-\sin{T_2})\cos{N_3} \\
\beta = \cos{T_1}(\sin{T_2}-\sin{T_3})\sin{N_1} + \cos{T_2}(\sin{T_3}-\sin{T_1}) \sin{N_2} +\cos{T_3}(\sin{T_1}-\sin{T_2})\sin{N_3}
\end{gather}
\end{subequations}


Figure \ref{fig:track} shows an example of the radius of curvature calculation for Hurricane Katrina. 




%\subsection{}     %% Appendix A1, A2, etc.


\noappendix       %% use this to mark the end of the appendix section. Otherwise the figures might be numbered incorrectly (e.g. 10 instead of 1).

%% Regarding figures and tables in appendices, the following two options are possible depending on your general handling of figures and tables in the manuscript environment:

%% Option 1: If you sorted all figures and tables into the sections of the text, please also sort the appendix figures and appendix tables into the respective appendix sections.
%% They will be correctly named automatically.

%% Option 2: If you put all figures after the reference list, please insert appendix tables and figures after the normal tables and figures.
%% To rename them correctly to A1, A2, etc., please add the following commands in front of them:



%% Please add \clearpage between each table and/or figure. Further guidelines on figures and tables can be found below.

\authorcontribution{TEXT} %% this section is mandatory

\competinginterests{TEXT} %% this section is mandatory even if you declare that no competing interests are present

\disclaimer{TEXT} %% optional section

\begin{acknowledgements}
This work was supported by NSF award 
\end{acknowledgements}


%% REFERENCES
\bibliographystyle{copernicus}
\bibliography{aiyyer_references.bib}
%%
%% URLs and DOIs can be entered in your BibTeX file as:
%%
%% URL = {http://www.xyz.org/~jones/idx_g.htm}
%% DOI = {10.5194/xyz}


%% LITERATURE CITATIONS
%%
%% command                        & example result
%% \citet{jones90}|               & Jones et al. (1990)
%% \citep{jones90}|               & (Jones et al., 1990)
%% \citep{jones90,jones93}|       & (Jones et al., 1990, 1993)
%% \citep[p.~32]{jones90}|        & (Jones et al., 1990, p.~32)
%% \citep[e.g.,][]{jones90}|      & (e.g., Jones et al., 1990)
%% \citep[e.g.,][p.~32]{jones90}| & (e.g., Jones et al., 1990, p.~32)
%% \citeauthor{jones90}|          & Jones et al.
%% \citeyear{jones90}|            & 1990



%% FIGURES

%% When figures and tables are placed at the end of the MS (article in one-column style), please add \clearpage
%% between bibliography and first table and/or figure as well as between each table and/or figure.

% The figure files should be labelled correctly with Arabic numerals (e.g. fig01.jpg, fig02.png).


\clearpage

%================================================================================================================
%================================================================================================================
%All Tables
%================================================================================================================


\begin{table*}[t]
  \caption{Speed (km hr$^{-1}$), tangential and curvature acceleration (km hr$^{-1}$ day$^{-1}$ ) of Atlantic TCs
   in the IBTRaCs database as a function of latitude. Storm instances classified as ET or NR have been excluded.
   N refers to number of 3-hourly track positions in each latitude-bin over the period 1966--2019. }
\begin{tabular}{|cc|ccc|ccc|ccc|}
\hline
& & \multicolumn{3}{|c|}{Speed} & \multicolumn{3}{c|}{Tang. Accel} & \multicolumn{3}{c|}{Curv. Accel}\\
\hline
Latitude & N & Mean & Median & Std Dev. & Mean & Median &  Std Dev.& Mean & Median & Std Dev.\\
	  5--15 &  5870 & 23.53 &  23.4 &   9.4 &  0.18 &   0.0 &  14.7 & 11.65 &   8.2 &  13.1 \\
	 10--20 & 13087 & 21.10 &  20.6 &   9.3 & -0.13 &  -0.0 &  15.3 & 12.29 &   8.5 &  13.3 \\
	 15--25 & 13656 & 18.66 &  18.1 &   8.6 &  0.26 &  -0.0 &  16.0 & 13.90 &   9.4 &  15.6 \\
	 20--30 & 13074 & 17.21 &  16.4 &   8.6 &  1.40 &   0.4 &  17.1 & 16.48 &  11.2 &  19.0 \\
	 25--35 & 13905 & 17.74 &  16.2 &  10.2 &  2.74 &   1.3 &  19.7 & 18.04 &  12.2 &  20.6 \\
	 30--40 & 10815 & 21.37 &  18.6 &  13.5 &  5.33 &   3.0 &  22.8 & 19.34 &  13.4 &  21.0 \\
	 35--45 &  5272 & 29.41 &  26.7 &  18.0 &  8.27 &   4.8 &  27.4 & 22.41 &  15.8 &  22.9 \\
	 40--50 &  1607 & 42.37 &  41.0 &  20.8 &  8.75 &   6.4 &  34.9 & 28.45 &  19.9 &  29.0 \\
	 45--55 &   329 & 55.69 &  53.9 &  19.4 &  6.81 &   6.3 &  37.8 & 36.50 &  25.9 &  38.9 \\
\bottomhline
\end{tabular}
%\belowtable{} % Table Footnotes
\label{tab:ATLSA}
\end{table*}
%==================================================================================


\begin{table*}[t]
  \caption{Trends in Speed (km hr$^{-1}$ year$^{-1}$). \emph{All storms} refers to all instances of a
    system recorded in the IBTraCs. ET refers to storm ntaure designated as extratropical, while NR refers to instances
    when the storm nature was not recorded. }

\begin{tabular}{|c|cccc|cccc|}
\hline
& \multicolumn{4}{|c|}{1966--2019}& \multicolumn{4}{|c|}{1949--2019}\\
& \multicolumn{2}{c}{LR} & \multicolumn{2}{c|}{MK-TS}& \multicolumn{2}{c}{LR} & \multicolumn{2}{c|}{MK-TS}\\
& Trend & P-value & Trend & P-value & Trend & P-value & Trend & P-value\\
\hline
Atlantic   &  0.029 & 0.19 &  0.028 & 0.15  & -0.016 & 0.28 & -0.016 & 0.32\\
(All storms) & & & & & & & &  \\
\hline
Atlantic  & -0.007 & 0.70 & -0.008 & 0.62 & -0.004 & 0.77 & -0.007 & 0.48\\
(Excluding ET,NR)  & & & & & & & &  \\
\hline
W. Pacific &  0.005 & 0.67 & -0.003 & 0.88 & \bf{-0.023} & < .01 & \bf{-0.027} & <.01\\
(All storms) & & & & & & & &  \\
\hline
\end{tabular}
%\belowtable{} % Table Footnotes
\label{tab:ATLSA}
\end{table*}

%==================================================================================
%==================================================================================


\begin{table*}[t]
  \caption{Trends in rapid tangential acceleration (km hr$^{-1}$ day$^{-1}$ year$^{-1}$) of Atlantic tropical cyclones. Storm instances  classified as ET or NR are excluded. Three cut-offs for defining rapid are used: Values exceeeding 90th, 85th and 60th percentile of all acceleration. }

\begin{tabular}{|c|cccc|cccc|cccc|}
\hline
%& \multicolumn{4}{|c|}{1966--2019}  & \multicolumn{4}{|c|}{1966--2019}\\
& \multicolumn{4}{|c|}{ 90th percentile } & \multicolumn{4}{|c|}{80th percentile }& \multicolumn{4}{|c|}{60th percentile}\\
Atlantic  & \multicolumn{2}{c}{LR} & \multicolumn{2}{c|}{MK-TS}& \multicolumn{2}{c}{LR} & \multicolumn{2}{c|}{MK-TS} & \multicolumn{2}{c}{LR} &  \multicolumn{2}{c|}{MK-TS}\\
(Excluding ET,NR) & Trend & P-value & Trend & P-value & Trend & P-value & Trend & P-value & Trend & P-value & Trend & P-value\\
\hline
Full basin  & -0.097 & 0.01 & -0.091 & 0.01 & -0.088 &  <.01 & -0.084 &  <.01 & -0.051 & 0.01 & -0.050 & 0.02 \\
	  5--25 & -0.094 & 0.00 & -0.102 & 0.00   & -0.081 & 0.00 & -0.080 & 0.00 & -0.046 & 0.01 & -0.047 & 0.02\\
	 10--30 & -0.080 & 0.01 & -0.082 & 0.00   & -0.079 & 0.00 & -0.072 & 0.00 & -0.048 & 0.01 & -0.049 & 0.01\\
	 15--35 & -0.086 & 0.02 & -0.079 & 0.01   & -0.084 & 0.00 & -0.079 & 0.00 & -0.043 & 0.03 & -0.045 & 0.02\\
	 20--40 & -0.113 & 0.01 & -0.110 & 0.01   & -0.088 & 0.01 & -0.086 & 0.02 & -0.046 & 0.06 & -0.051 & 0.06\\
	 25--45 & -0.099 & 0.08 & -0.090 & 0.05   & -0.075 & 0.10 & -0.065 & 0.11 & -0.061 & 0.07 & -0.047 & 0.10\\
	 30--50 & -0.124 & 0.06 & -0.119 & 0.04   & -0.097 & 0.08 & -0.099 & 0.11 & -0.078 & 0.08 & -0.061 & 0.18\\
\hline
\hline
\end{tabular}
%\belowtable{} % Table Footnotes
\label{tab:ATLAC}
\end{table*}




%==================================================================================
%==================================================================================


\begin{table*}[t]
  \caption{Trends in rapid tangential acceleration (km hr$^{-1}$ day$^{-1}$ year$^{-1}$) of Atlantic tropical cyclones. All storm instance, including classified as NR and ET are considered here. Three cut-offs for defining rapid are used: Values exceeeding 90th, 85th and 60th percentile of all acceleration. }

\begin{tabular}{|c|cccc|cccc|cccc|}
\hline
%& \multicolumn{4}{|c|}{1966--2019}  & \multicolumn{4}{|c|}{1966--2019}\\
& \multicolumn{4}{|c|}{ 90th percentile } & \multicolumn{4}{|c|}{80th percentile }& \multicolumn{4}{|c|}{60th percentile}\\
Atlantic  & \multicolumn{2}{c}{LR} & \multicolumn{2}{c|}{MK-TS}& \multicolumn{2}{c}{LR} & \multicolumn{2}{c|}{MK-TS} & \multicolumn{2}{c}{LR} &  \multicolumn{2}{c|}{MK-TS}\\
(Excluding ET,NR) & Trend & P-value & Trend & P-value & Trend & P-value & Trend & P-value & Trend & P-value & Trend & P-value\\
\hline
Full basin &  -0.075 & 0.06 & -0.080 & 0.05 & -0.055 & 0.05 & -0.055 & 0.06  & -0.026 & 0.22 & -0.024 & 0.32\\


\hline
\hline
\end{tabular}
%\belowtable{} % Table Footnotes
\label{tab:ATLAC}
\end{table*}


%================================================================================================================
%All Figures
%================================================================================================================



%========================================================
%Figure 1
%========================================================

\begin{figure*}[t]
\includegraphics[width=7cm]{katrina.png}
\caption{Illustration of the circle-fit and radius of curvature calculations at
five selected locations along the track of hurricane Katrina (2005).}\label{fig:track}
\end{figure*}
%========================================================


\clearpage


%========================================================
%Figure 1
%========================================================

\begin{figure*}[t]
  \centering
    \includegraphics[width=.9\textwidth]{distribution_NA.png}
    \caption{Distribution of (a) Speed; (b) Tangential acceleration and (c) Curvature acceleration of
     Atlantic TCs as a function of latitude. Data from 1966--2019 was binned within 10$^o$-wide overlapping
     latitude bins. Statistics shown are: median (horizontal line within the box), mean (dot), and 10th,
     25th, 75th and 90th percentiles.}
  \label{fig:distNA}
\end{figure*}


%========================================================



\clearpage


%========================================================
%Figure 
%========================================================

\begin{figure*}[t]
  \includegraphics[width=7cm]{atl_speed_trend.png}
  \includegraphics[width=7cm]{wpa_speed_trend.png}

\caption{Annual-mean speed of storms in the (a) North Atlantic; and (b) western North Pacific, along with the linear trend. The purple curve is for all storms while the green curve excludes instances when
the storm was classified as ET or NR.}\label{fig:atl_speed}
\end{figure*}
%========================================================


\clearpage

%========================================================
%Figure 
%========================================================

\begin{figure*}[t]
  \includegraphics[width=7cm]{atl_tang_no_etnr.png}
   \includegraphics[width=7cm]{atl_tang_20_40_no_etnr.png}

  \caption{Annual-mean tangential acceleration exceeding thresholds of 60th, 80th and 90th percentile values for (a) Entire Atlantic; and (b) Latitude range 20-40$^o$N. Instances of storms classified as NR and ET were excluded. The linear trend for each timeseries is shown by the straight line.}\label{fig:atl_speed}
\end{figure*}
%========================================================



%%% TABLES
%%%
%%% The different columns must be seperated with a & command and should
%%% end with \\ to identify the column brake.
%
%%% ONE-COLUMN TABLE
%
%%t
%\begin{table}[t]
%\caption{TEXT}
%\begin{tabular}{column = lcr}
%\tophline
%
%\middlehline
%
%\bottomhline
%\end{tabular}
%\belowtable{} % Table Footnotes
%\end{table}
%
%%% TWO-COLUMN TABLE
%
%%t
%\begin{table*}[t]
%\caption{TEXT}
%\begin{tabular}{column = lcr}
%\tophline
%
%\middlehline
%
%\bottomhline
%\end{tabular}
%\belowtable{} % Table Footnotes
%\end{table*}
%
%%% LANDSCAPE TABLE
%
%%t
%\begin{sidewaystable*}[t]
%\caption{TEXT}
%\begin{tabular}{column = lcr}
%\tophline
%
%\middlehline
%
%\bottomhline
%\end{tabular}
%\belowtable{} % Table Footnotes
%\end{sidewaystable*}
%
%
%%% MATHEMATICAL EXPRESSIONS
%
%%% All papers typeset by Copernicus Publications follow the math typesetting regulations
%%% given by the IUPAC Green Book (IUPAC: Quantities, Units and Symbols in Physical Chemistry,
%%% 2nd Edn., Blackwell Science, available at: http://old.iupac.org/publications/books/gbook/green_book_2ed.pdf, 1993).
%%%
%%% Physical quantities/variables are typeset in italic font (t for time, T for Temperature)
%%% Indices which are not defined are typeset in italic font (x, y, z, a, b, c)
%%% Items/objects which are defined are typeset in roman font (Car A, Car B)
%%% Descriptions/specifications which are defined by itself are typeset in roman font (abs, rel, ref, tot, net, ice)
%%% Abbreviations from 2 letters are typeset in roman font (RH, LAI)
%%% Vectors are identified in bold italic font using \vec{x}
%%% Matrices are identified in bold roman font
%%% Multiplication signs are typeset using the LaTeX commands \times (for vector products, grids, and exponential notations) or \cdot
%%% The character * should not be applied as mutliplication sign
%
%
%%% EQUATIONS
%
%%% Single-row equation
%
%\begin{equation}
%
%\end{equation}
%
%%% Multiline equation
%
%\begin{align}
%& 3 + 5 = 8\\
%& 3 + 5 = 8\\
%& 3 + 5 = 8
%\end{align}
%
%
%%% MATRICES
%
%\begin{matrix}
%x & y & z\\
%x & y & z\\
%x & y & z\\
%\end{matrix}
%
%
%%% ALGORITHM
%
%\begin{algorithm}
%\caption{...}
%\label{a1}
%\begin{algorithmic}
%...
%\end{algorithmic}
%\end{algorithm}
%
%
%%% CHEMICAL FORMULAS AND REACTIONS
%
%%% For formulas embedded in the text, please use \chem{}
%
%%% The reaction environment creates labels including the letter R, i.e. (R1), (R2), etc.
%
%\begin{reaction}
%%% \rightarrow should be used for normal (one-way) chemical reactions
%%% \rightleftharpoons should be used for equilibria
%%% \leftrightarrow should be used for resonance structures
%\end{reaction}
%
%
%%% PHYSICAL UNITS
%%%
%%% Please use \unit{} and apply the exponential notation


\clearpage
\appendixfigures  %% needs to be added in front of appendix figures


%\begin{figure}[ht]
%  \centering
%    \includegraphics[width=.7\textwidth]{distributionWP.png}
%  \caption{}
%  \label{fig:distWP}
%\end{figure}


\appendixtables   %% needs to be added in front of appendix tables


\begin{table*}[t]
\caption{Speed (km hr$^{-1}$), tangential and curvature acceleration  (km hr$^{-1}$ day$^{-1}$ ) of all Western North Pacific non-ET storms as a function of latitude: N refers to number of 3-hourly track positions in each latitude-bin over the period 1966--2019. Storm positions corresponding to 
nature labels "ET" and "NR" were excluded.}

\begin{tabular}{cc|ccc|ccc|ccc}
\tophline
& & \multicolumn{3}{|c|}{Speed} & \multicolumn{3}{c|}{Tang. Accel} & \multicolumn{3}{c}{Curv. Accel}\\
\hline
       Latitude &     N &  Mean & Median & Std Dev. & Mean & Median &  Std Dev.& Mean & Median & Std Dev.\\
	  5--15 & 30731 & 17.78 &  17.1 &   8.2 &  0.21 &   0.2 &  16.4 & 15.16 &   9.8 &  18.0 \\
	 10--20 & 47338 & 17.10 &  16.4 &   7.9 &  0.14 &   0.1 &  16.3 & 15.23 &   9.9 &  17.9 \\
	 15--25 & 45753 & 16.99 &  16.1 &   8.4 &  1.10 &   0.7 &  17.6 & 16.18 &  10.5 &  18.8 \\
	 20--30 & 30724 & 18.96 &  17.4 &  10.4 &  2.99 &   2.0 &  20.1 & 17.83 &  11.8 &  19.7 \\
	 25--35 & 17931 & 23.28 &  20.2 &  14.1 &  5.92 &   3.6 &  24.4 & 19.90 &  14.0 &  20.0 \\
	 30--40 &  9655 & 30.28 &  26.1 &  18.6 &  8.36 &   5.4 &  30.4 & 23.92 &  17.2 &  23.3 \\
	 35--45 &  4274 & 39.44 &  36.1 &  21.2 &  5.86 &   4.6 &  39.1 & 30.96 &  21.8 &  30.8 \\
	 40--50 &  1556 & 43.48 &  39.9 &  23.1 & -1.44 &  -0.9 &  47.3 & 38.24 &  26.5 &  37.2 \\
	 45--55 &   346 & 47.69 &  40.9 &  27.6 & -6.33 &  -3.1 &  51.3 & 41.47 &  29.3 &  39.1 \\
\hline
\end{tabular}
%\belowtable{} % Table Footnotes
\label{tab:WPSA}
\end{table*}




\end{document}
