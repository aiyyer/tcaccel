%% Copernicus Publications Manuscript Preparation Template for LaTeX Submissions
%% ---------------------------------
%% This template should be used for copernicus.cls
%% The class file and some style files are bundled in the Copernicus Latex Package, which can be downloaded from the different journal webpages.
%% For further assistance please contact Copernicus Publications at: production@copernicus.org
%% https://publications.copernicus.org/for_authors/manuscript_preparation.html


%% Please use the following documentclass and journal abbreviations for preprints and final revised papers.

%% 2-column papers and preprints
\documentclass[wcd, manuscript]{copernicus}
% Weather and Climate Dynamics (wcd)


%% \usepackage commands included in the copernicus.cls:
%\usepackage[english]{babel}
%\usepackage{tabularx}
%\usepackage{cancel}
%\usepackage{multirow}
%\usepackage{supertabular}
%\usepackage{algorithmic}
%\usepackage{algorithm}
\usepackage{amsthm}
\usepackage{float}
\usepackage{subfig}
\usepackage{rotating}


\begin{document}

\title{Along-Track Acceleration of Tropical Cyclones: A Proxy For Extratropical Interactions}


% \Author[1]{given_name}{surname}

\Author[1]{Anantha}{Aiyyer}
\Author[1]{Terrell}{Wade}

\affil[1]{Department of Marine, Earth and Atmospheric Sciences, North Carolina State University}


%% The [] brackets identify the author with the corresponding affiliation. 1, 2, 3, etc. should be inserted.

%% If an author is deceased, please mark the respective author name(s) with a dagger, e.g. "\Author[2,$\dag$]{Anton}{Aman}", and add a further "\affil[$\dag$]{deceased, 1 July 2019}".

%% If authors contributed equally, please mark the respective author names with an asterisk, e.g. "\Author[2,*]{Anton}{Aman}" and "\Author[3,*]{Bradley}{Bman}" and add a further affiliation: "\affil[*]{These authors contributed equally to this work.}".


\correspondence{NAME (EMAIL)}

\runningtitle{TEXT}

\runningauthor{TEXT}





\received{}
\pubdiscuss{} %% only important for two-stage journals
\revised{}
\accepted{}
\published{}

%% These dates will be inserted by Copernicus Publications during the typesetting process.


\firstpage{1}

\maketitle



\begin{abstract}
  Forecasters have long known that rapid changes in tropical cyclones (TC) motion occur during interaction with extratropical wavetrains. While TC speed has received much attention in the past, acceleration has not. Here, using a large data sample, we formally examine the synoptic-scale patterns associated with  tangential and normal components of TC acceleration. During periods of rapid acceleration, the large-scale flow consists of a developing wavepacket in the extratropics, with the storm straddled between an upstream trough and downstream ridge. This pattern is consistent with the process of extratropical transition (ET) of TCs. In contrast, during near-zero or rapid deceleration, the extratropical wavepacket remains poleward, with a ridge  located directly north of the storm center.  On average, the tangential acceleration peaks about 18 hours prior to the ET designation. The curvature acceleration increases before ET and stabilizes thereafter.
%
 We also find a robust decrease in the annually averaged values of rapid acceleration over the past 38 years within the latitude band most commonly associated with ET over the Atlantic. Consequently, we hypothesize that trends in rapid acceleration can be used as an independent metric to assess changes in the extratropical stormtrack activity that mediate ET.
\end{abstract}


%\copyrightstatement{TEXT}


\introduction  %% \introduction[modified heading if necessary]
A tropical cyclone (TC), to a leading order, can abstracted as a nearly isolated vortex embedded in an external flow. From this view, its motion is primarily controlled by the large scale steering current. Secondary effects on TC motion involve interactions with the background vertical wind shear and potential vorticity gradients. The former influences the generation of new convection and redevelopment of the TC. The latter relates to Rossby wave dispersion that leads to the formation of the so-called $\beta$-gyres and nonlinear advection. As noted by \cite{E2018}, steady progress has been made in our understanding of TC motion and this is reflected in our ability to forecast their tracks. 


\section{data}


We used the IBTrACS v4 \citep{KnappIBTRaCS} data for track information. We did not apply a threshold for maximum sustained winds, but discarded all storms that lasted less than two days. Selected atmospheric fields are taken from the the European Center for Medium Range Weather Forecasting's (ECMWF) ERA-Interim reanalysis \citep[ERAI; ][]{DUS11}. To ascertain whether a tropical storm underwent ET, we use the \emph{nature} designation in the IBTraCS database which relies on the judgement of the forecasters of the agency responsible for the ocean basin. For the North Atlantic basin, this designation made by the US national hurricane center (NHC) is available for the entire period considered here. While compiling ET statistics, we only consider cases where the storm nature designation switched from TS to ET at least once during its lifetime. 


\subsection{HEADING}
TEXT


\subsubsection{Tangential and curvature acceleration}
At any location along a TC track, the acceleration of a fluid element moving with the center of a TC can be written as:

\begin{equation}
    \mathbf{a} = \frac{dV}{dt} \mathbf{\hat{s}} + \frac{V^2}{R} \mathbf{\hat{n}} 
    \label{eq:acc}
\end{equation}

Where V is the forward speed and R is the radius of curvature of the track at that location. Here $\mathbf{\hat{s}}$ and $\mathbf{\hat{n}}$ are unit vectors in this \emph{natural coordinate} system. The former is directed along the TC motion. The latter is normal to the local tangent and aligned along the radius of curvature.  The first term in equation \ref{eq:acc} is the tangential acceleration and the second is the curvature or normal acceleration. The tangential speed V at any track location (given by index j)  is calculated as:


\conclusions  %% \conclusions[modified heading if necessary]
TEXT

%% The following commands are for the statements about the availability of data sets and/or software code corresponding to the manuscript.
%% It is strongly recommended to make use of these sections in case data sets and/or software code have been part of your research the article is based on.

\codeavailability{TEXT} %% use this section when having only software code available


\dataavailability{TEXT} %% use this section when having only data sets available


\codedataavailability{TEXT} %% use this section when having data sets and software code available


\sampleavailability{TEXT} %% use this section when having geoscientific samples available


\videosupplement{TEXT} %% use this section when having video supplements available


\appendix
\section{}    %% Appendix A

\subsection{}     %% Appendix A1, A2, etc.


\noappendix       %% use this to mark the end of the appendix section. Otherwise the figures might be numbered incorrectly (e.g. 10 instead of 1).

%% Regarding figures and tables in appendices, the following two options are possible depending on your general handling of figures and tables in the manuscript environment:

%% Option 1: If you sorted all figures and tables into the sections of the text, please also sort the appendix figures and appendix tables into the respective appendix sections.
%% They will be correctly named automatically.

%% Option 2: If you put all figures after the reference list, please insert appendix tables and figures after the normal tables and figures.
%% To rename them correctly to A1, A2, etc., please add the following commands in front of them:

\appendixfigures  %% needs to be added in front of appendix figures

\appendixtables   %% needs to be added in front of appendix tables

%% Please add \clearpage between each table and/or figure. Further guidelines on figures and tables can be found below.

\authorcontribution{TEXT} %% this section is mandatory

\competinginterests{TEXT} %% this section is mandatory even if you declare that no competing interests are present

\disclaimer{TEXT} %% optional section

\begin{acknowledgements}
This work was supported by NSF award 
\end{acknowledgements}




%% REFERENCES

%% The reference list is compiled as follows:

%\begin{thebibliography}{}
%\bibitem[AUTHOR(YEAR)]{LABEL1}
%REFERENCE 1
%\bibitem[AUTHOR(YEAR)]{LABEL2}
%REFERENCE 2
%\end{thebibliography}

%% Since the Copernicus LaTeX package includes the BibTeX style file copernicus.bst,
%% authors experienced with BibTeX only have to include the following two lines:
%%
\bibliographystyle{copernicus}
\bibliography{aiyyer_references.bib}
%%
%% URLs and DOIs can be entered in your BibTeX file as:
%%
%% URL = {http://www.xyz.org/~jones/idx_g.htm}
%% DOI = {10.5194/xyz}


%% LITERATURE CITATIONS
%%
%% command                        & example result
%% \citet{jones90}|               & Jones et al. (1990)
%% \citep{jones90}|               & (Jones et al., 1990)
%% \citep{jones90,jones93}|       & (Jones et al., 1990, 1993)
%% \citep[p.~32]{jones90}|        & (Jones et al., 1990, p.~32)
%% \citep[e.g.,][]{jones90}|      & (e.g., Jones et al., 1990)
%% \citep[e.g.,][p.~32]{jones90}| & (e.g., Jones et al., 1990, p.~32)
%% \citeauthor{jones90}|          & Jones et al.
%% \citeyear{jones90}|            & 1990



%% FIGURES

%% When figures and tables are placed at the end of the MS (article in one-column style), please add \clearpage
%% between bibliography and first table and/or figure as well as between each table and/or figure.

% The figure files should be labelled correctly with Arabic numerals (e.g. fig01.jpg, fig02.png).


%% ONE-COLUMN FIGURES

%%f
%\begin{figure}[t]
%\includegraphics[width=8.3cm]{track_rad_example.png}
%\caption{TEXT}
%\end{figure}


%
%%% TWO-COLUMN FIGURES
%
%%f
\begin{figure*}[t]
\includegraphics[width=12cm]{track_rad_example.png}
 \caption{Illustration of the circle-fit and radius of curvature at four selected locations along the track of hurricane Katrina (2005).}\label{fig:track}
\end{figure*}
%
%
%%% TABLES
%%%
%%% The different columns must be seperated with a & command and should
%%% end with \\ to identify the column brake.
%
%%% ONE-COLUMN TABLE
%
%%t
%\begin{table}[t]
%\caption{TEXT}
%\begin{tabular}{column = lcr}
%\tophline
%
%\middlehline
%
%\bottomhline
%\end{tabular}
%\belowtable{} % Table Footnotes
%\end{table}
%
%%% TWO-COLUMN TABLE
%
%%t
%\begin{table*}[t]
%\caption{TEXT}
%\begin{tabular}{column = lcr}
%\tophline
%
%\middlehline
%
%\bottomhline
%\end{tabular}
%\belowtable{} % Table Footnotes
%\end{table*}
%
%%% LANDSCAPE TABLE
%
%%t
%\begin{sidewaystable*}[t]
%\caption{TEXT}
%\begin{tabular}{column = lcr}
%\tophline
%
%\middlehline
%
%\bottomhline
%\end{tabular}
%\belowtable{} % Table Footnotes
%\end{sidewaystable*}
%
%
%%% MATHEMATICAL EXPRESSIONS
%
%%% All papers typeset by Copernicus Publications follow the math typesetting regulations
%%% given by the IUPAC Green Book (IUPAC: Quantities, Units and Symbols in Physical Chemistry,
%%% 2nd Edn., Blackwell Science, available at: http://old.iupac.org/publications/books/gbook/green_book_2ed.pdf, 1993).
%%%
%%% Physical quantities/variables are typeset in italic font (t for time, T for Temperature)
%%% Indices which are not defined are typeset in italic font (x, y, z, a, b, c)
%%% Items/objects which are defined are typeset in roman font (Car A, Car B)
%%% Descriptions/specifications which are defined by itself are typeset in roman font (abs, rel, ref, tot, net, ice)
%%% Abbreviations from 2 letters are typeset in roman font (RH, LAI)
%%% Vectors are identified in bold italic font using \vec{x}
%%% Matrices are identified in bold roman font
%%% Multiplication signs are typeset using the LaTeX commands \times (for vector products, grids, and exponential notations) or \cdot
%%% The character * should not be applied as mutliplication sign
%
%
%%% EQUATIONS
%
%%% Single-row equation
%
%\begin{equation}
%
%\end{equation}
%
%%% Multiline equation
%
%\begin{align}
%& 3 + 5 = 8\\
%& 3 + 5 = 8\\
%& 3 + 5 = 8
%\end{align}
%
%
%%% MATRICES
%
%\begin{matrix}
%x & y & z\\
%x & y & z\\
%x & y & z\\
%\end{matrix}
%
%
%%% ALGORITHM
%
%\begin{algorithm}
%\caption{...}
%\label{a1}
%\begin{algorithmic}
%...
%\end{algorithmic}
%\end{algorithm}
%
%
%%% CHEMICAL FORMULAS AND REACTIONS
%
%%% For formulas embedded in the text, please use \chem{}
%
%%% The reaction environment creates labels including the letter R, i.e. (R1), (R2), etc.
%
%\begin{reaction}
%%% \rightarrow should be used for normal (one-way) chemical reactions
%%% \rightleftharpoons should be used for equilibria
%%% \leftrightarrow should be used for resonance structures
%\end{reaction}
%
%
%%% PHYSICAL UNITS
%%%
%%% Please use \unit{} and apply the exponential notation


\end{document}
